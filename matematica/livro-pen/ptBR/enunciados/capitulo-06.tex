\chapter{Equações Funcionais}

\quoting{A única maneira de aprender Matemática é fazendo Matemática.}{Paul
  Halmos}

\textbf{F 1. } Prove que existe uma função $f$ do conjunto dos naturais para o
conjunto dos naturais tal que $f(f(n))=n^2$ para todo $n \in
\mathbb{N}$. \source{Singapore 1996}

\textbf{F 2. } Encontre todas as funções sobrejetoras
$f:\mathbb{N} \to \mathbb{N}$ tais que para todos $m,n\in \mathbb{N}$:
$$ m \vert n \Longleftrightarrow f(m) \vert f(n).$$ \source{Turkey 1995}

\textbf{F 3. } Encontre todas as funções $f:\mathbb{N} \to \mathbb{N}$ tais que
para todos $n\in \mathbb{N}$: $$ f(n+1) > f(f(n)). $$ \source{IMO 1977/6}

\textbf{F 4. } Encontre todas as funções $f:\mathbb{N} \to \mathbb{N}$ tais que
para todo $n\in \mathbb{N}$: $$ f(f(f(n)))+f(f(n))+f(n)=3n. $$ \source{}

\textbf{F 5. } Encontre todas as funções $f:\mathbb{N} \to \mathbb{N}$ tais que
para todo $n\in \mathbb{N}$: $$ f(f(m)+f(n))=m+n. $$ \source{}

\textbf{F 6. } Encontre todas as funções $f:\mathbb{N} \to \mathbb{N}$ tais que
para todo $n\in \mathbb{N}$: $$ f^{(19)}(n)+97f(n)=98n+232. $$ \source{IMO
  unused 1997}

\textbf{F 7. } Encontre todas as funções $f:\mathbb{N} \to \mathbb{N}$ tais que
para todo $n\in \mathbb{N}$: $$ f(f(n))+f(n)=2n+2001 \text{ or } 2n+2002. $$
\source{Balkan 2002}

\textbf{F 8. } Encontre todas as funções $f:\mathbb{N} \to \mathbb{N}$ tais que
para todo $n\in \mathbb{N}$: $$ f(f(f(n)))+6f(n)=3f(f(n))+4n+2001. $$
\source{USAMO Summer Program 2001}

\textbf{F 9. } Encontre todas as funções
$f:\mathbb{N}_{0} \rightarrow \mathbb{N}_{0}$ tais que para todo
$n\in \mathbb{N}_0$: $$ f(f(n))+f(n)=2n+6. $$ \source{Austria 1989}

\textbf{F 10. } Encontre todas as funções $f:\mathbb{N}_{0} \to \mathbb{N}_{0}$
tais que para todo $n\in \mathbb{N}_0$: $$ f(m+f(n))=f(f(m))+f(n). $$
\source{IMO 1996/3}

\textbf{F 11. } Encontre todas as funções $f:\mathbb{N}_{0} \to \mathbb{N}_{0}$
tais que para todo $m,n\in \mathbb{N}_0$: $$ mf(n)+nf(m)=(m+n)f(m^2 +n^2). $$
\source{Canada 2002}

\textbf{F 12. } Encontre todas as funções $f:\mathbb{N} \to \mathbb{N}$ tais que
para todo $m,n\in \mathbb{N}$:

\begin{itemize}
\item $f(2)=2$,
\item $f(mn)=f(m)f(n)$,
\item $f(n+1)>f(n)$.
\end{itemize} \source{Canada 1969}

\textbf{F 13. } Encontre todas as funções $f:\mathbb{Z} \to \mathbb{Z}$ tais que
para todo $m\in \mathbb{Z}$: $$f(f(m))=m+1.$$ \source{Slovenia 1997}

\textbf{F 14. } Encontre todas as funções $f:\mathbb{Z} \to \mathbb{Z}$ tais que
para todo $m\in\mathbb{Z}$:
\begin{itemize}
\item $f(m+8) \le f(m)+8$,
\item $f(m+11) \ge f(m)+11$.
\end{itemize} \source{}

\textbf{F 15. } Encontre todas as funções $f:\mathbb{Z} \to \mathbb{Z}$ tais que
para todo $m,n\in \mathbb{Z}$: $$ f(m+f(n))=f(m)-n. $$ \source{APMC 1997}

\textbf{F 16. } Encontre todas as funções $f:\mathbb{Z} \to \mathbb{Z}$ tais que
para todo $m,n\in \mathbb{Z}$: $$ f(m+f(n)) = f(m)+n. $$ \source{South Africa
  1997}

\textbf{F 17. } Encontre todas as funções $h:\mathbb{Z} \to \mathbb{Z}$ tais que
para todo $x,y\in \mathbb{Z}$: $$ h(x+y)+h(xy)=h(x)h(y)+1.$$ \source{Belarus
  1999}

\textbf{F 18. } Encontre todas as funções $f:\mathbb{Q} \to \mathbb{R}$ tais que
para todo $x,y\in \mathbb{Q}$: $$ f(xy)=f(x)f(y)-f(x+y)+1.$$ \source{APMC 1984}

\textbf{F 19. } Encontre todas as funções $f:\mathbb{Q}^{+} \to \mathbb{Q}^{+}$
tais que para todo $x,y \in \mathbb{Q}^{+}$:
$$ f \left( x + \frac{y}{x} \right) =f(x) +\frac{f(y)}{f(x)} +2y, \; x,y \in
\mathbb{Q}^{+}. $$ \source{}

\textbf{F 20. } Encontre todas as funções $f:\mathbb{Q} \to \mathbb{Q}$ tais que
para todo $x,y \in \mathbb{Q}$: $$ f(x+y)+f(x-y)=2(f(x)+f(y)).$$ \source{Nordic
  Mathematics Contest 1998}

\textbf{F 21. } Encontre todas as funções $f,g,h:\mathbb{Q} \to \mathbb{Q}$ tais
que para todo $x,y \in \mathbb{Q}$: $$ f(x+g(y))=g(h(f(x)))+y.$$ \source{KMO
  Winter Program Test 2001}

\textbf{F 22. } Encontre todas as funções $f:\mathbb{Q}^{+} \to \mathbb{Q}^{+}$
tais que para todo $x\in \mathbb{Q}^+$:

\begin{itemize}
\item $f(x+1)=f(x)+1$,
\item $f(x^2)=f(x)^2$.
\end{itemize} \source{Ukrine 1997}

\textbf{F 23. } Seja ${\mathbb Q}^+$ o conjunto dos racionais
positivos. Construa uma função $f: {\mathbb Q}^+ \rightarrow {\mathbb Q}^+$ tal
que $$f(xf(y)) = \frac{f(x)}{y}$$ for all $x, y \in {\mathbb Q}^+$. \source{IMO
  1990/4}

\textbf{F 24. } A função $f$ é definida nos inteiros positivos por
$$\left\{
  \begin{array}{rcl}
    f(1) &=& 1, \\
    f(3) &=& 3, \\
    f(2n) &=& f(n), \\
    f(4n + 1) &=& 2f(2n + 1) - f(n), \\
    f(4n + 3) &=& 3f(2n + 1) - 2f(n),
  \end{array}
\right.$$

para todos os inteiros positivos $n$. Determine o número de inteiros positivos
$n$, menores que ou iguais a $1988$, para os quais $f(n) = n$. \source{IMO
  1988/3}

\textbf{F 25. } Considere todas as funções $f:\mathbb{N}\to\mathbb{N}$
satisfazendo $f(t^2 f(s)) = s(f(t))^2$ para todos $s$ e $t$ em $N$. Determine o
menor valor possível de $f(1998)$. \source{IMO 1998/6}

\textbf{F 26. } A função $f:\mathbb{N}\to\mathbb{N}_0$ satisfaz para todo
$m,n\in\mathbb{N}$:
$$ f(m+n)-f(m)-f(n)=0\text{ ou }1, \; f(2)=0, \; f(3)>0, \; \text{ e }
f(9999)=3333.$$ Determine $f(1982)$. \source{IMO 1982/1}

\textbf{F 27. } Encontre todas as funções $f: \mathbb{N} \to \mathbb{N}$ tais
que para todo $m,n\in \mathbb{N}$: $$ f(f(m)+f(n))=m+n.$$ \source{IMO Short List
  1988}

\textbf{F 28. } Encontre todas as funções sobrejetoras
$f: \mathbb{N} \to \mathbb{N}$ tais que para todo $n\in \mathbb{N}$:
$$ f(n) \ge n+(-1)^{n}.$$ \source{Romania 1986}

\textbf{F 29. } Encontre todas as funções
$f: \mathbb{Z}\setminus\{0\} \to \mathbb{Q}$ tais que para todo
$x,y \in \mathbb{Z}\setminus\{0\}$:
$$ f \left( \frac{x+y}{3} \right) =\frac{f(x)+f(y)}{2}, \; \; x, y \in
\mathbb{Z}\setminus\{0\} $$ \source{Iran 1995}

\textbf{F 30. } Encontre todas as funções estritamente crescentes
$f: \mathbb{N} \to \mathbb{N}$ tais que $$ f(f(n))=3n. $$ \source{}

\textbf{F 31. } Encontre todas as funções $f: \mathbb{Z}^{2} \to \mathbb{R}^{+}$
tais que para todo $i, j \in \mathbb{Z}$:
$$ f(i,j)=\frac{f(i+1, j)+f(i,j+1)+f(i-1,j)+f(i,j-1)}{4}.$$ \source{}

\textbf{F 32. } Encontre todas as funções $f: \mathbb{Q} \to \mathbb{Q}$ tais
que para todo $x,y,z \in \mathbb{Q}$:
$$ f(x+y+z)+f(x-y)+f(y-z)+f(z-x)=3f(x)+3f(y)+3f(z).$$ \source{}

\textbf{F 33. } Mostre que existe uma função bijetora
$f:\mathbb{N}_{0} \to \mathbb{N}_{0}$ tal que para todo $m,n\in
\mathbb{N}_0$: $$ f(3mn+m+n)=4f(m)f(n)+f(m)+f(n).$$ \source{IMO ShortList 1996}

%%% Local Variables:
%%% mode: latex
%%% coding: utf-8-unix
%%% fill-column: 80
%%% TeX-master: "MASTER"
%%% End:
