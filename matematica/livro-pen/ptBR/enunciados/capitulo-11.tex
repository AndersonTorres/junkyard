\chapter{Teoria Analítica dos Números}

\quoting{Teoristas dos Números são como devoradores de lótus - tendo
  experimentado esta comida, eles nunca a abandonam.}{Leopold Kronecker}

\textbf{K 1. } Seja $\alpha$ a raiz positiva da equação $x^{2}=1991x+1$. Para
números naturais $m$ e $n$ defina
$$ m*n=mn+\lfloor\alpha m \rfloor \lfloor \alpha n\rfloor. $$ Prove que para
todos os números naturais $p$, $q$, e $r$, $$ (p*q)*r=p*(q*r). $$ \source{IMO
  ShortList 1991 P20 (IRE 3)}

\textbf{K 2. } Prove que para qualquer inteiro positivo $n$,
$$ \left\lfloor \frac{n}{3} \right\rfloor + \left\lfloor \frac{n+2}{6}
\right\rfloor + \left\lfloor \frac{n+4}{6} \right\rfloor = \left\lfloor
  \frac{n}{2} \right\rfloor + \left\lfloor \frac{n+3}{6} \right\rfloor.  $$
\source{[EbMk, pp. 5]}

\textbf{K 3. } Prove que para qualquer inteiro positivo $n$,
$$ \left\lfloor \frac{n+1}{2} \right\rfloor + \left\lfloor \frac{n+2}{4}
\right\rfloor + \left\lfloor \frac{n+4}{8} \right\rfloor + \left\lfloor
  \frac{n+8}{16} \right\rfloor + \cdots = n. $$ \source{[EbMk, pp. 7]}

\textbf{K 4. } Mostre que para todos os inteiros positivos $n$,
$$ \lfloor \sqrt{n}+\sqrt{n+1}\rfloor =\lfloor \sqrt{4n+1}\rfloor =\lfloor
\sqrt{4n+2}\rfloor =\lfloor \sqrt{4n+3}\rfloor. $$ \source{Canada 1987}

\textbf{K 5. } Encontre todos os números reais $\alpha$ para os quais a
igualdade
$$ \lfloor \sqrt{n}+\sqrt{n+ \alpha}\rfloor =\lfloor \sqrt{4n+1}\rfloor $$ vale
para todos os inteiros positivos $n$. \source{CRUX, Problem 1650, Iliya Bluskov}

\textbf{K 6. } Prove que para todos os inteiros positivos $n$,
$$ \lfloor \sqrt{n}+\sqrt{n+1}+\sqrt{n+2}\rfloor =\lfloor \sqrt{9n+8}\rfloor. $$
\source{Iran 1996}

\textbf{K 7. } Prove que para todos os inteiros positivos $n$,
$$ \lfloor \sqrt[3]{n}+\sqrt[3]{n+1}\rfloor =\lfloor \sqrt[3]{8n+3}\rfloor.$$
\source{MM, Problem 1410, Seung-Jin Bang}

\textbf{K 8. } Prove que
$\lfloor \sqrt[3]{n}+\sqrt[3]{n+1}+\sqrt[3]{n+2}\rfloor =\lfloor
\sqrt[3]{27n+26}\rfloor$ para todos os inteiros positivos
$n$. \source{Can. Math. Soc. Notes, Problem P11, Mih\'aly Bencze}

\textbf{K 9. } Mostre que para todos os inteiros positivos $m$ e $n$,
$$ \gcd(m, n) = m+n-mn+2\sum^{m-1}_{k=0} \left \lfloor \frac{kn}{m} \right
\rfloor. $$ \source{Taiwan 1998}

\textbf{K 10. } Mostre que para todos os primos $p$,
$$ \sum^{p-1}_{k=1} \left \lfloor \frac{k^3}{p} \right \rfloor
=\frac{(p+1)(p-1)(p-2)}{4}. $$ \source{AMM, Problem 10346, David Doster}

\textbf{K 11. } Seja $p$ um número primo da forma $4k+1$. Mostre que
$$ \sum^{p-1}_{i=1} \left( \left \lfloor \frac{2i^2}{p} \right \rfloor - 2\left
    \lfloor \frac{i^2}{p} \right \rfloor \right) = \frac{p-1}{2}. $$
\source{Korea 2000}

\textbf{K 12. } Seja $p=4k+1$ um primo. Mostre que
$$ \sum^{k}_{i=1} \left \lfloor \sqrt{ ip } \right \rfloor = \frac{p^2
  -1}{12}. $$ \source{[IHH pp.142]}

\textbf{K 13. } Suponha que $n \ge 2$. Prove que
$$ \sum_{k=2}^{n} \left\lfloor \frac{n^2}{k} \right\rfloor = \sum_{k=n+1}^{n^2}
\left\lfloor \frac{n^2}{k} \right\rfloor.  $$ \source{CRUX, Problem 2321, David
  Doster}

\textbf{K 14. } Seja $a, b, n$ inteiros positivos com $\gcd(a, n)=1$. Prove
que $$ \sum_{k} \left\{ \frac{ak+b}{n} \right\}=\frac{n-1}{2}, $$ onde $k$
percorre o sistema completo de resíduos módulo $m$. \source{}

\textbf{K 15. } Encontre o número total de valores inteiros que a função
$$f(x) = \lfloor x\rfloor + \lfloor 2x\rfloor + \left\lfloor
  \frac{5x}{3}\right\rfloor + \lfloor 3x\rfloor + \lfloor 4x\rfloor $$ toma para
números reais $x$ com $0 \leq x \leq 100$. \source{APMO 1993/2}

\textbf{K 16. } Prove ou disprove que exista um real positivo $u$ tal que
$\lfloor u^n \rfloor -n$ é um inteiro para todo inteiro positivo $n$. \source{}

\textbf{K 17. } Determine todos os números reais $a$ tais que
$$ 4\lfloor an\rfloor =n+\lfloor a\lfloor an\rfloor \rfloor \; \text{for all} \;
n \in \mathbb{N}. $$ \source{Bulgaria 2003}

\textbf{K 18. } Existem irracionais $a, b>1$ e
$\lfloor a^{m}\rfloor \not=\lfloor b^{n}\rfloor $ para quaisquer inteiros
positivos $m$ e $n$? \source{[Tt] Tournament of the Towns 2002 Spring/A-Level}

\textbf{K 19. } Sejam $a, b, c$, e $d$ números reais. Suponha que
$\lfloor na\rfloor +\lfloor nb\rfloor =\lfloor nc\rfloor +\lfloor nd\rfloor $
para todos os inteiros positivos $n$. Mostre que pelo menos um dos $a+b$, $a-c$,
$a-d$ é inteiro. \source{[PbAw, pp. 5]}

\textbf{K 20. } Encontre todas as soluções inteiras de
$$ \left\lfloor \frac{x}{1!} \right\rfloor + \left\lfloor \frac{x}{2!}
\right\rfloor + \cdots + \left\lfloor \frac{x}{10!} \right\rfloor =1001. $$
\source{(S. Reznichenko) [Ams, pp. 45]}

%%% Local Variables:
%%% mode: latex
%%% coding: utf-8-unix
%%% fill-column: 80
%%% TeX-master: "MASTER"
%%% End:
