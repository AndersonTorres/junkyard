\chapter{Números Primos e Compostos}

\quoting{Onde há números, há beleza.}{Proclus Diadochus}

\textbf{C 1. } Prove que o número $512^{3} +675^{3}+ 720^{3}$ é
composto. \source{[DfAk, pp. 50] Leningrad Mathematical Olympiad 1991}

\textbf{C 2. } Sejam $a, b, c, d$ inteiros com $a>b>c>d>0$. Suponha que
$ac+bd=(b+d+a-c)(b+d-a+c)$. Prove que $ab+cd$ não é primo. \source{IMO 2001/6}

\textbf{C 3. } Encontre a soma de todos os divisores positivos distintos do
número $104060401$. \source{MM, Problem Q614, Rod Cooper}

\textbf{C 4. } Prove que $1280000401$ é composto. \source{}

\textbf{C 5. } Prove que $\frac{5^{125}-1}{5^{25}-1}$ é um número composto.
\source{IMO Short List 1992 P16}

\textbf{C 6. } Encontre um fator de $2^{33}-2^{19}-2^{17}-1$ que se situa entre
$1000$ e $5000$. \source{MM, Problem Q684, Noam Elkies}

\textbf{C 7. } Mostre que existe um inteiro positivo $k$ tal que
$k \cdot 2^{n}+1$ é composto para todo $n \in \mathbb{N}_{0}$. \source{USA 1982}

\textbf{C 8. } Mostre que para todo inteiro $k>1$, existem infinitos números
naturais $n$ tais que $k \cdot 2^{2^n} + 1$ é composto. \source{[VsAs]}

\textbf{C 9. } Quatro inteiros são marcados em um círculo. Em cada passo nós
substituímos simultaneamente cada número pela diferença entre este número e o
próximo no círculo em uma dada direção (isto é, os números $a$, $b$, $c$, $d$
são substituídos por $a-b$, $b-c$, $c-d$, $d-a$). É possível após $1996$ destes
passos termos números $a$, $b$, $c$ e $d$ tais que os números $|bc-ad|$,
$|ac-bd|$ e $|ab-cd|$ são primos? \source{IMO Short List 1996 N1}

\textbf{C 10. } Represente o número $989 \cdot 1001 \cdot 1007 +320$ como um
produto de primos. \source{[DfAk, pp. 9] Leningrad Mathematical Olympiad 1987}

\textbf{C 11. } Em 1772 Euler descobriu o curioso fato que $n^2 +n+41$ é primo
quando $n$ é qualquer um de $0,1,2, \cdots, 39$. Mostre que existem $40$ valores
inteiros de $n$ para os quais este polinômio \textit{não é primo}. \source{[JDS,
  pp. 26]}

\textbf{C 12. } Mostre que existem infinitos números primos. \source{}

\textbf{C 13. } Encontre todos os números naturais $n$ para os quais todo número
natural cuja representação decimal tem $n-1$ dígitos $1$ e um dígito $7$ é
primo. \source{IMO Short List 1990 USS1}

\textbf{C 14. } Prove que não existem polinômios $P$ e $Q$ tais que
$$ \pi(x)=\frac{P(x)}{Q(x)} $$ para todo $x \in \mathbb{N}$. \source{[Tma,
  pp. 101]}

\textbf{C 15. } Mostre que existem dois quadrados consecutivos tais que existem
pelo menos $1000$ primos entre eles. \source{MM, Problem Q789, Norman
  Schaumberger}

\textbf{C 16. } Prove que para todo primo $p$ no intervalo
$\left]n, \frac{4n}{3} \right]$, $p$ é divisor de
$$ \sum^{n}_{j=0} {\binom{n}{j}}^4. $$ \source{MM, Problem 1392, George Andrews}

\textbf{C 17. } Sejam $a$, $b$, e $n$ inteiros positivos com $\gcd (a,b)=1$. Sem
usar o Teorema de Dirichlet, mostre que existem infinitos $k \in \mathbb{N}$
tais que $\gcd(ak+b, n)=1$. \source{[AaJc pp.212]}

\textbf{C 18. } Sem usar o Teorema de Dirichlet, mostre que existem infinitos
números primos terminados com o dígito $9$. \source{}

\textbf{C 19. } Seja $p$ um primo ímpar. Sem usar o Teorema de Dirichlet, mostre
que existem infinitos primos da forma $2pk+1$. \source{[AaJc pp.176]}

\textbf{C 20. } Verifique que, para cada $r \ge 1$, existem infinitos primos $p$
com $p \equiv 1 \; \pmod{2^r}$. \source{[GjJj pp.140]}

\textbf{C 21. } Prove que se $p$ é um primo, então $p^{p}-1$ tem um fator primo
que é congruente a $1$ módulo $p$. \source{[Ns pp.176]}

\textbf{C 22. } Seja $p$ um número primo. Prove que existe um número primo $q$
tal que para todo inteiro $n$, $n^p -p$ não é múltiplo de $q$. \source{IMO
  2003/6}

\textbf{C 23. } Sejam $p_{1}=2, p_{2}={3}, p_{3}=5, \cdots, p_{n}$ os primeiros
$n$ números primos, onde $n \ge 3$. Prove que
$$ \frac{1}{{p_1}^2}+\frac{1}{{p_2}^2}+\cdots+\frac{1}{{p_n}^2}+\frac{1}{p_1 p_2
  \cdots p_n} < \frac{1}{2}. $$ \source{Yugoslavia 2001}

\textbf{C 24. } Seja $p_n$ o $n$th número primo. Mostre que a série infinita
$$ \sum^{\infty}_{n=1} \frac{1}{p_n} $$ diverge. \source{}

\textbf{C 25. } Prove que para cada inteiro positivo $n$, existem $n$ inteiros
positivos consecutivos, nenhum dos quais potência inteira de um número primo.
\source{IMO 1989/5}

\textbf{C 26. } Mostre que $n^{\pi(2n)-\pi(n)}<4^{n}$ para todo inteiro positivo
$n$. \source{[GjJj pp.36]}

\textbf{C 27. } Seja $s_n$ a soma dos primeiros $n$ primos. Prove que para cada
$n$ existe um inteiro cujo quadrado está entre $s_n$ e $s_{n+1}$. \source{[Tma, pp. 102]}

\textbf{C 28. } Dado um inteiro ímpar $n>3$, sejam $k$ e $t$ os menores inteiros
positivos tais que ambos $kn+1$ e $tn$ são quadrados. Prove que $n$ é primo se e
somente se ambos $k$ e $t$ são maiores que $\frac{n}{4}$ \source{[Tma, pp. 128]}

\textbf{C 29. } Seja $n \ge 5$ um inteiro. Mostre que $n$ é primo se e somente
se $n_{i} n_{j} \neq n_{p} n_{q}$ para toda partição de $n$ em $4$ inteiros,
$n=n_{1}+n_{2}+n_{3}+n_{4}$, e para cada permutação $(i, j, p, q)$ de
$(1, 2, 3, 4)$. \source{Singapore 1989}

\textbf{C 30. } Prove que não existem inteiros positivos  $a$ e $b$ tais que
para todos os primos distintos $p$ e $q$ maiores que $1000$, o número $ap+bq$
também é primo. \source{Russia 1996}

\textbf{C 31. } Seja $p_{n}$ o $n$ - ésimo número primo. Para todo $n \ge 6$,
prove que $$ \pi \left( \sqrt{p_1 p_2 \cdots p_n} \right) > 2n. $$ \source{[Rh,
  pp. 43]}

\textbf{C 32. } Existe um bloco de $1000$ inteiros positivos consecutivos não
contendo nenhum número primo, a saber, $1001!+2$, $1001!+3$, $\cdots$,
$1001!+1001$. Existe um bloco de $1000$ inteiros positivos consecutivos contendo
exatamente cinco números primos? \source{[Tt] Tournament of the Towns 2001
  Fall/O-Level}

\textbf{C 33. } Prove que existem infinitos números primos gêmeosse e somente se
existem infinitos inteiros que não podem ser escritos em nenhuma das formas a
seguir:
$$ 6uv+u+v, \;\; 6uv+u-v, \;\; 6uv-u+v, \;\; 6uv-u-v, $$ para $u$ e $v$ inteiros
positivos. \source{[PeJs, pp. 160], S. Golomb}

\textbf{C 34. } É sabido que sempre existe um primo entre $n$ e $2n-7$ para todo
$n \ge 10$. Prove que, com exceção de $1$, $4$, e $6$, todo número natural pode
ser escrito como soma de primos distintos. \source{[PeJs, pp. 174]}

\textbf{C 35. } Prove que não existem onze primos, todos eles menores que
$20000$, que formam uma progressão aritmética. \source{[DNI, 19]}

%%% Local Variables:
%%% mode: latex
%%% coding: utf-8-unix
%%% fill-column: 80
%%% TeX-master: "MASTER"
%%% End:
