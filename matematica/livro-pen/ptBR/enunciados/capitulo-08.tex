\chapter{Teoria Aditiva dos Números}

\quoting{Eu lembro de uma vez de ter ido vê-lo quando ele estava convalescente
  em Putney. Eu embarquei num táxi de número $1729$ e notei que o número me
  parecia meio entediante, e que eu esperava que não fosse um mau augúrio.
  `Não,' respondera ele, `este é um número bastante interessante; é o menor
  número que pode ser expresso como soma de dois cubos de duas formas
  distintas.'}{G. H. Hardy, sobre Ramanujan}

\textbf{H 1. } Mostre que qualquer inteiro pode ser expressado como soma de dois
quadrados e um cubo. \source{AMM, Problem 10426, Noam Elkies and Irving
  Kaplansky}

\textbf{H 2. } Prove que infinitos inteiros positivos não podem  ser escritos na
forma
$$ {x_{1}}^3+ {x_{2}}^5+ {x_{3}}^7+ {x_{4}}^9+ {x_{5}}^{11}, $$ onde $x_{1},
x_{2}, x_{3}, x_{4}, x_{5} \in \mathbb{N}$.  \source{Belarus 2002, V. Bernik}

\textbf{H 3. } Determine todos os inteiros positivos que são expressáveis na
forma $$ a^2 +b^2 +c^2 +c, $$ onde $a$, $b$, $c$ são inteiros. \source{MM,
  Problem Q817, Robert B. McNeill}

\textbf{H 4. } Mostre que qualquer racional positivo pode ser representado como
soma de três cubos racionais positivos. \source{}

\textbf{H 5. } Mostre que todo inteiro maior que $1$ pode ser escrito como soma
de dois inteiros livres de quadrados. \source{[IHH, pp. 474]}

\textbf{H 6. } Prove que todo inteiro $n \ge 12$ é a soma de dois números
compostos. \source{[Tma, pp. 22]}

\textbf{H 7. } Prove que qualquer inteiro positivo pode ser representado como um
agregado de potências diferentes de $3$, os termos desse agregado sendo
combinados pelos símbolos $+$ e $-$ escolhidos apropriadamente. \source{[Rdc
  pp.24]}

\textbf{H 8. } O inteiro $9$ pode ser escrito como soma de dois inteiros
consecutivos: $9=4+5$. Mais que isso, ele pode ser escrito como soma de inteiros
positivos consecutivos (mais de um) de exatamente duas maneiras, a saber,
$9=4+5=2+3+4$. Existe um inteiro que pode ser escrito como a soma de $1990$
inteiros consecutivos e que pode ser escrito como soma de inteiros positivos
consecutivos (mais de um) de exatamente $1990$ maneiras? \source{IMO Short List
  1990 AUS3}

\textbf{H 9. } Para cada inteiro positivo $\,n,\;S(n)\,$ é definido como o maior
inteiro tal que, para todo inteiro positivo $\,k\leq S(n),\;n^{2}\,$ pode ser
escrito como soma de $\,k\,$ quadrados positivos.

\begin{enumerate}
\item Prove que $S(n)\leq n^{2}-14$ para cada $n\geq 4$.
\item Encontre um inteiro $n$ tal que $S(n)=n^{2}-14$.
\item Prove que existem infinitos inteiros $n$ tais que
  $S(n)=n^{2}-14$.
\end{enumerate} \source{IMO 1992/6}

\textbf{H 10. } Para cada inteiro positivo $n$, $f(n)$ denota o número de
maneiras de representar $n$ como soma de potências de dois com expoentes
inteiros não-negativos. Representações que diferem somente na ordem de suas
parcelas são consideradas a mesma. Por exemplo, $f(4)=4$, porque o número $4$
pode ser representado das seguintes quatro formas:
$$4, 2+2, 2+1+1, 1+1+1+1. $$
Prove que, para cada inteiro positivo $n \geq 3$,
$$ 2^{n^2/4} < f(2^n) < 2^{n^2/2}. $$
\source{IMO 1997/6}

\textbf{H 11. } A função positiva $p(n)$ é definida como o número de formas que
o inteiro positivo $n$ pode ser escrito como soma de inteiros positivos. Por
exemplo, $5=4+1=3+2=3+1+1=2+2+1=2+1+1+1=1+1+1+1+1$ nos dá $p(5)=7$. Mostre que,
para todos os inteiros positivos $n \ge 2$,
$$ 2^{\lfloor \sqrt{n}\rfloor} \le p(n) \le n^{3 \lfloor\sqrt{n}\rfloor }. $$
\source{[Hua pp.199]}

\textbf{H 12. } Seja $a_{1}=1$, $a_{2}=2$, $a_3$, $a_4$, $\cdots$ a sequência
dos inteiros positivos da forma $2^{\alpha} 3^{\beta}$, onde $\alpha$ e $\beta$
são inteiros não-negativos. Prove que todo inteiro positivo é expressável na
forma $$ a_{i_{1}}+a_{i_{2}}+ \cdots + a_{i_{n}}, $$ onde nenhum somando é
múltiplo de qualquer outro. \source{MM, Problem Q814, Paul Erd\"os, further used
  as Putnam 2005 A1}

\textbf{H 13. } Seja $n$  um inteiro não-negativo. Encontre todos os inteiros
não-negativos $a$, $b$, $c$, $d$ tais que
$$ a^2 +b^2 +c^2 +d^2 = 7 \cdot 4^{n}. $$ \source{Romania 2001, Laurentiu
  Panaitopol}

\textbf{H 14. } Encontre todos os inteiros $m>1$ tais que $m^3$ é a soma de $m$
quadrados de inteiros consecutivos. \source{AMM, Problem E3064, Ion Cucurezeanu}

\textbf{H 15. } Prove que existem infinitos inteiros $n$ tais que $n, n+1, n+2$
são, cada um deles, a soma dos quadrados de dois inteiros. \source{Putnam 2000}

\textbf{H 16. } Seja $p$ um número primo da forma $4k+1$. Suponha que $r$ é um
resíduo quadrático de $p$ e $s$ é um resíduo não-quadrático de $p$. Mostre que
$p=a^2 +b^2$, onde
$$ a=\frac{1}{2} \sum^{p-1}_{i=1} \left( \frac{i(i^2 -r)}{p} \right),
b=\frac{1}{2} \sum^{p-1}_{i=1} \left( \frac{i(i^2 -s)}{p} \right). $$
Aqui, $\left( \frac{k}{p} \right)$ denota o Símbolo de Legendre.
\source{Jacobsthal}

\textbf{H 17. } Seja $p$ um primo com $p \equiv 1 \pmod{4}$. Seja $a$ o único
inteiro tal que
$$ p=a^2 +b^2, \; a \equiv -1 \pmod{4}, \; b \equiv 0 \; \pmod{2} $$
Prove que
$$ \sum^{p-1}_{i=0} \left( \frac{i^3 +6i^2 +i }{p} \right) = 2 \left(
  \frac{2}{p} \right), $$ where $\left(\frac kp\right)$ denota o Símbolo de
Legendre. \source{AMM, Problem 2760, Kenneth S. Williams}

\textbf{H 18. } Seja $n$ um inteiro da forma $a^2 + b^2$, onde $a$ e $b$ são
primos entre si e tais que se $p$ é um primo, $p \leq \sqrt{n}$, então $p$ é
divisor de $ab$. Determine todos os tais $n$. \source{APMO 1994/3}

\textbf{H 19. } Se um inteiro $n$ é tal que $7n$ é da forma $a^2 +3b^2$, prove
que $n$ também é da mesma forma. \source{India 1998}

\textbf{H 20. } Seja $A$ o conjunto de inteiros positivos da forma $a^2 +2b^2$,
onde $a$ e $b$ são inteiros e $b \neq 0$. Mostre que se $p$ é um número primo e
$p^2 \in A$, então $p \in A$. \source{Romania 1997, Marcel Tena}

\textbf{H 21. } Mostre que um inteiro pode ser expresso como diferença de dois
quadrados se e somente se não é da forma $4k+2 \; (k \in \mathbb{Z})$. \source{}

\textbf{H 22. } Mostre que qualquer inteiro pode ser expresso na forma
$a^{2}+b^{2}-c^{2}$, onde $a, b, c \in \mathbb{Z}$. \source{}

\textbf{H 23. } Sejam $a$ e $b$ inteiros positivos com $\gcd(a, b)=1$. Mostre
que cada inteiro maior que $ab-a-b$ pode ser expresso na forma $ax+by$, onde
$x, y \in \mathbb{N}_{0}$. \source{}

\textbf{H 24. } Sejam $a, b$ e $c$ inteiros positivos, nunca dois deles tendo
divisor comum maior que $1$. Mostre que $2abc-ab-bc-ca$ é o maior inteiro que
não pode ser expresso na forma $xbc+yca+zab$, onde $x, y, z \in \mathbb{N}_{0}$
\source{IMO 1983/3}

\textbf{H 25. } Determine, com prova, o maior número que é o produto de inteiros
positivos cuja soma é $1976$. \source{IMO 1976/4}

\textbf{H 26. } Prove que qualquer inteiro positivo pode ser representado como a
soma de números de Fibonacci, nunca dois deles consecutivos. \source{Zeckendorf}

\textbf{H 27. } Mostre que o conjunto dos inteiros positivos que não podem ser
representados como soma de quadrados perfeitos distintos é finito. \source{IMO
  Short List 2000 N6}

\textbf{H 28. } Seja $a_{1}, a_{2}, a_{3}, \cdots$ uma sequência crescente de
inteiros não-negativos tal que cada inteiro não-negativo pode ser expresso
unicamente na forma $a_{i}+2a_{j}+4a_{k}$, onde $i, j, $ e $k$ não são
necessariamente distintos. Determine $a_{1998}$. \source{IMO Short List 1998
  P21}

\textbf{H 29. } Uma sequência finita de inteiros $a_{0}, a_{1}, \cdots, a_{n}$ é
chamada \textit{quadrática} se para cada $i \in \{1,2,\cdots,n \}$ temos a
igualdade $\vert a_{i}-a_{i-1} \vert = i^2$.

\begin{enumerate}
\item Prove que para quaisquer dois inteiros $b$ e $c$, existe um número natural
  $n$ e uma sequência quadrática com $a_{0}=b$ e $a_{n}=c$.
\item Encontre o menor número natural $n$ para o qual existe uma sequência
  quadrática com $a_{0}=0$ e $a_{n}=1996$.
\end{enumerate} \source{IMO Short List 1996 N3}

\textbf{H 30. } Um inteiro positivo composto é um produto $ab$ com $a$ e $b$
inteiros não necessariamente distintos em $\{2,3,4,\dots\}$. Mostre que todo
inteiro positivo composto é expressável como $xy+xz+yz+1$, com $x,y,z$ inteiros
positivos. \source{Putnam 1988/B1}

\textbf{H 31. } Sejam $a_{1}, a_{2}, \cdots, a_{k}$ inteiros positivos primos
entre si. Determine o maior número que não poe ser expresso na forma
$$ x_{1} a_{2} a_{3} \cdots a_{k} + x_{2} a_{1} a_{3} \cdots a_{k} + \cdots +
x_{k} a_{1} a_{2} \cdots a_{k-1} $$
para alguns inteiros não-negativos $x_{1}, x_{2}, \cdots, x_{k}$. \source{MM,
  Problem 1561, Emre Alkan}

\textbf{H 32. } Se $n$ é um inteiro positivo que pode ser expresso na forma
$n=a^{2}+b^{2}+c^{2}$, onde $a, b, c$ são inteiros positivos, prove que para
cada inteiro positivo $k$, $n^{2k}$ pode ser expresso na forma $A^2 +B^2 +C^2$,
onde $A, B, C$ são inteiros positivos. \source{[KhKw, pp. 21]}

\textbf{H 33. } Prove que cada inteiro positivo que não é membro do conjunto
infinito abaixo é igual à soma de dois ou mais números distintos do conjunto
$$ \{ 3, -2, 2^2 3, -2^3, \cdots, 2^{2k} 3, -2^{2k+1}, \cdots \}=
\{3, -2, 12, -8, 48, -32, 192, \cdots \}. $$ \source{[EbMk, pp. 46]}

\textbf{H 34. } Sejam $k$ e $s$ inteiros positivos ímpares tais que
$$ \sqrt{3k-2} -1 \le s \le \sqrt{4k}. $$ Mostre que existem inteiros
não-negativos $t$, $u$, $v$, e $w$ tais que
$$ k=t^{2}+u^{2}+v^{2}+w^{2}, \;\; \text{e} \;\; s=t+u+v+w. $$ \source{[Wsa,
  pp. 271]}

\textbf{H 35. } Seja $S_{n}=\{1,n,n^{2},n^{3}, \cdots \}$, onde $n$ é um inteiro
maior que $1$. Encontre o menor número $k=k(n)$ tal que exista um número que
possa ser expresso como soma de $k$ elementos (possivelmente repetidos) de
$S_{n}$ de mais de uma maneira. (Rearranjos são considerados idênticos.)
\source{[GML, pp. 37]}

\textbf{H 36. } Encontre o menor valor de $n$ para o qual existem inteiros
$x_{1}$, $x_{2}$, $\cdots$, $x_{n}$ tais que cada inteiro entre $1000$ e $2000$
(inclusive) pode ser escrito como a soma (sem repetição) de um ou mais dos
inteiros $x_{1}$, $x_{2}$, $\cdots$, $x_{n}$. \source{[GML, pp. 144]}

\textbf{H 37. } De quantas maneiras $2^{n}$ pode ser expresso como soma de
quatro quadrados de números naturais? \source{[DNI, 28]}

\textbf{H 38. } Mostre que
\begin{enumerate}
\item infinitos quadrados perfeitos são a soma de um quadrado perfeito e um
  número primo,
\item infinitos quadrados perfeitos não são a soma de um quadrado perfeito e um
  número primo.
\end{enumerate} \source{[JDS, pp. 25]}

\textbf{H 39. } A famosa conjectura de Goldbach é a asserção que todo inteiro
par maior que $2$ é a soma de dois primos. Exceto $2$, $4$, e $6$, todo inteiro
par é soma de inteiros positivos compostos: $n=4+(n-4)$. Qual é o maior inteiro
positivo par que não é soma de dois inteiros ímpares compostos?\source{[JDS,
  pp. 25]}

\textbf{H 40. } Prove que para cada inteiro positivo $K$ existem infinitos
inteiros positivos pares que podem ser escritos de mais de $K$ formas como soma
de dois primos ímpares. \source{MM, Feb. 1986, Problem 1207, Barry Powell}

\textbf{H 41. } Um inteiro positivo $n$ é abundante se a soma de seus divisores
próprios excede $n$. Mostre que todo inteiro maior que $89 \times 315$ é a soma
de dois números abundantes. \source{MM, Nov. 1982, Problem 1130,
  J. L. Selfridge}

\textbf{H 42. } Encontre todos os números naturais de três dígitos que são
iguais à terceira potência da soma de seus dígitos. \source{}

\textbf{H 43. } Prove que todos os inteiros positivos podem ser escritos como
diferença de dois inteiros positivos que têm o mesmo número de divisores
primos. \source{}

%%% Local Variables:
%%% mode: latex
%%% coding: utf-8-unix
%%% fill-column: 80
%%% TeX-master: "MASTER"
%%% End:
