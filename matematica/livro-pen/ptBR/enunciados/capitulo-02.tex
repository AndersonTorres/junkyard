\chapter{Congruências}

\quoting{Matemática é a rainha das ciências e Teoria dos Números é a rainha da
  Matemática.}{Johann Carl Friedrich Gauss}

\textbf{B 1. } Se $p$ é um primo ímpar, prove que
$$ \binom{k}{p} \equiv \left\lfloor \frac{k}{p} \right\rfloor \pmod{p}. $$
\source{[Tma, pp. 127]}

\textbf{B 2. } Suponha que $p$ é um primo ímpar. Prove que
$$ \sum_{j=0}^p \binom{p}{j} \binom{p+j}{j} \equiv 2^p + 1\pmod{p^2}. $$
\source{Putnam 1991/B4}

\textbf{B 3. } Mostre que
$$ (-1)^{\frac{p-1}{2}}\binom{p-1}{\frac{p-1}{2}} \equiv 4^{p-1} \pmod{p^3} $$
para todos os números primos $p$ com $p \ge 5$.  \source{Morley}

\textbf{B 4. } Seja $n$ um inteiros positivo. Prove que $n$ é primo se e somente
se $$ \binom{n-1}{k} \equiv (-1)^{k} \pmod{n} $$ para todos
$k \in \{ 0, 1, \cdots, n-1 \}$.  \source{MM, Problem 1494, Emeric Deutsch and
  Ira M.Gessel}

\textbf{B 5. } Prove que para $n\geq 2$,
$$ \underbrace{2^{2^{\cdots^{2}}}}_{n\text{ terms}} \equiv
\underbrace{2^{2^{\cdots^{2}}}}_{n-1\text{ terms}} \; \pmod{n}. $$
\source{Putnam 1997/B5}

\textbf{B 6. } Mostre que, para qualquer inteiro fixo $\,n \geq 1,\,$ a
sequência $$ 2, \; 2^2, \; 2^{2^2}, \; 2^{2^{2^2}}, \cdots \pmod{n} $$ é
eventualmente constante. \source{USA 1991}

\textbf{B 7. } Alguém incorretamente lembrou-se do Pequeno Teorema de Fermat
como dizendo que a congruência $a^{n+1} \equiv a \; \pmod{n}$ é válida para todo
$a$ se $n$ é primo. Descreva o conjunto de inteiros $n$ para os quais esta
propriedade é de fato verdadeira. \source{[DZ] posed by Don Zagier at the St
  AndrewsColloquium 1996}

\textbf{B 8. } Caracterize o conjunto de inteiros positivos $n$ tais que para
todo inteiro $a$ a sequência $a$, $a^2$, $a^3$, $\cdots$ é periódica módulo $n$.
\source{MM Problem Q889, Michael McGeachie and StanWagon}

\textbf{B 9. } Mostre que existe um número composto $n$ tal que
$a^n \equiv a \; \pmod{n}$ para todo $a \in \mathbb{Z}$. \source{}

\textbf{B 10. } Seja $p$ um número primo da forma $4k+1$. Suponha que $2p+1$ é
primo. Mostre que não existe $k \in \mathbb{N}$ com $k<2p$ e
$2^k \equiv 1 \; \pmod{2p+1}$. \source{}

\textbf{B 11. } Durante um intervalo, $n$ crianças numa escola sentam-se em
círculo em volta do seu professor para brincar de um jogo. O professor caminha
no sentido horário próximo às crianças e dá doces a algumas delas de acordo com
a regra a seguir. Ele seleciona uma criança e dá a ela um doce, então salta a
próxima criança e dá um doce à próxima, então salta duas crianças e dá um doce à
próxima, então salta três e assim por diante. Determine os valores de $n$ para
os quais eventualmente, possivelmente após muitas rodadas, todas as crianças
terão pelo menos um doce cada. \source{APMO 1991/4}

\textbf{B 12. } Suponha que $m>2$, e seja $P$ o produto dos inteiros positivos
menores que $m$ que são primos com $m$. Mostre que $P \equiv -1 \pmod{m}$ se
$m=4$, $p^n$, ou $2p^{n}$, onde $p$ é um primo ímpar, e  $P \equiv 1 \pmod{m}$
caso contrário. \source{[AaJc, pp. 139]}

\textbf{B 13. } Seja $\Gamma$ consistindo de todos os polinômios em $x$ com
coeficientes inteiros. Para $f$ e $g$ em $\Gamma$ m $m$ um inteiro positivo,
seja $f \equiv g \pmod{m}$ implicando que todo coeficiente de $f-g$ é um
múltiplo inteiro de $m$. Seja $n$ e $p$ inteiros positivos com $p$ primo. Dado
que $f,g,h,r$ e $s$ estão em $\Gamma$ com $rf+sg\equiv 1 \pmod{p}$ e
$fg \equiv h \pmod{p}$, prove que existem $F$ e $G$ em $\Gamma$ com
$F \equiv f \pmod{p}$, $G \equiv g \pmod{p}$, e $FG \equiv h
\pmod{p^n}$. \source{Putnam 1986/B3}

\textbf{B 14. } Determine o número de inteiros $n \ge 2$ para os quais a
congruência $$ x^{25} \equiv x \; \pmod{n} $$ é verdadeira para todos os
inteiros $x$. \source{Bulgaria 1995}

\textbf{B 15. } Sejam $n_{1}, \cdots, n_{k}$  $a$ inteiros positivos que
satisfazem as seguintes condições:

\begin{itemize}
\item para todo $i \neq j$, $(n_{i}, n_{j})=1$,
\item para todo $i$, $a^{n_{i}} \equiv 1 \pmod{n_i}$,
\item para todo $i$, $n_{i}$ não é divisor de $a-1$. 
\end{itemize}

Mostre que existem pelo menos $2^{k+1}-2$ inteiros $x>1$ com
$a^{x} \equiv 1 \pmod{x}$. \source{Turkey 1993}

\textbf{B 16. } Determine todos os inteiros positivos $n \ge 2$ que satisfazem a
seguinte condição: Para todos os inteiros $a, b$ primos com $n$,
$$ a \equiv b \; \pmod{n} \Longleftrightarrow ab \equiv 1 \; \pmod{n}. $$
\source{IMO Short List 2000 N1}

\textbf{B 17. } Determine todos os inteiros positivos $n$ tais que
$ xy+1 \equiv 0 \; \pmod{n} $ implica $ x+y \equiv 0 \; \pmod{n}$.  \source{AMM,
  Problem S9, M. S. Klamkin and A.Liu}

\textbf{B 18. } Seja $p$ um número primo. Determine o grau maximal de um
polinômio $T(x)$ cujos coeficientes pertencem a $\{ 0,1,\cdots,p-1 \}$, cujo
grau é menor que $p$, e que satisfaz
$$ T(n)=T(m) \; \pmod{p} \Longrightarrow n=m \; \pmod{p} $$ para todos os
inteiros $n, m$. \source{Turkey 2000}

\textbf{B 19. } Sejam $a_{1}$, $\cdots$, $a_{k}$ e $m_{1}$, $\cdots$, $m_{k}$
inteiros com $2 \le m_1$ e $2m_{i} \le m_{i+1}$ for $1 \le i \le k-1$. Mostre
que existem infinitos inteiros $x$ que não satisfazem nenhuma das
congruências
$$ x \equiv a_{1} \; \pmod{m_{1}}, x \equiv a_{2} \; \pmod{m_{2}}, \cdots, x
\equiv a_{k} \; \pmod{m_{k}}. $$ \source{Turkey 1995}

\textbf{B 20. } Mostre que $1994$ é divisor de
$10^{900}-2^{1000}$. \source{Belarus 1994}

\textbf{B 21. } Determine os últimos três dígitos de $$ 2003^{2002^{2001}}. $$
\source{Canada 2003}

\textbf{B 22. } Prove que $1980^{1981^{1982}} + 1982^{1981^{1980}}$ é múltiplo
de $1981^{1981}$. \source{China 1981}

\textbf{B 23. } Seja $p$ um primo ímpar da forma $p=4n+1$. 

\begin{enumerate}
\item Mostre que $n$ é um resíduo quadrático $\pmod{p}$.
\item Calcule o valor $n^{n}$  $\pmod{p}$.
\end{enumerate} \source{}

\textbf{B 24. } Existem 16 números de três dígitos, usando somente três dígitos
diferentes, tais que cada um deixa um resto diferente módulo 16? \source{}

%%% Local Variables:
%%% mode: latex
%%% coding: utf-8-unix
%%% fill-column: 80
%%% TeX-master: "MASTER"
%%% End:
