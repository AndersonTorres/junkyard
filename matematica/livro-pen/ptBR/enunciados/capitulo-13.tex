\chapter{Problemas Miscelâneos}

\quoting{A Matemática ainda não está pronta para tais problemas.}{Paul Erd\"os}

\textbf{M 1. } São escolhidos dois inteiros positivos. A soma é revelada ao
lógico A, a soma dos quadrados é revelada ao lógico B. Ambos A e B recebem esta
informação e a informação contida nesta sentença. A conversação entre A e B
prossegue assim:

B inicia.

\begin{quote}
  B: ` Não posso dizer quais são.' \\
  A: ` Não posso dizer quais são.' \\
  B: ` Não posso dizer quais são.' \\
  A: ` Não posso dizer quais são.' \\
  B: ` Não posso dizer quais são.' \\
  A: ` Não posso dizer quais são.' \\
  B: ` Agora eu posso dizer quais são.'
\end{quote}

\begin{enumerate}
\item Quais são os dois números?
\item Quando B diz primeiro que ele não pode dizer quais são os números, A
  recebe um grande montante de informação. Mas quando A diz primeiro que ele não
  pode dizer quais são os dois números, B já sabe que A não pode saber quais são
  os números. Quão bem faz a B ouvir A?
\end{enumerate} \source{MM, May 1984, Problem 1173, Thomas S.Ferguson}

\textbf{M 2. } É dado que $2^{333}$ é um número de 101 dígitos cujo primeiro
dígito é 1. Quantos números $2^k$, $1 \le k \le 332$, têm o primeiro dígito 4?
\source{[Tt] Tournament of the Towns 2001 Fall/A-Level}

\textbf{M 3. } Existe uma potência de dois tal que é possível rearranjar os
dígitos e obter outra potência de dois? \source{[Pt] Tournament of the Towns}

\textbf{M 4. } Se $x$ é um número real tal que $x^2 -x$ é inteiro, e para algum
$n \ge 3$, $x^n -x$ também é inteiro, prove que $x$ é um inteiro.
\source{Ireland 1998}

\textbf{M 5. } Suponha que ambos $x^{3}-x$ e $x^{4}-x$ são inteiros para algum
número real $x$. Mostre que $x$ é inteiro. \source{Vietnam 2003 (Tran Nam Dung)}

\textbf{M 6. } Suponha que $x$ e $y$ são números complexos tais que
$$ \frac{x^n -y^n}{x-y} $$ são inteiros para certos quatro valores inteiros
positivos consecutivos de $n$. Prove que esta expressão também é inteira para
todos os inteiros positivos $n$. \source{AMM, Problem E2998, Clark Kimberling}

\textbf{M 7. } Seja $n$ um inteiro positivo. Mostre que
$$ \sum^{n}_{k=1} \tan^{2} \frac{k \pi}{2n+1} $$ é um inteiro ímpar. \source{}

\textbf{M 8. }  O conjunto $S = \{1/r: r=1,2,3,\ldots\}$ contém progressões
aritméticas de vários tamanhos. Por exemplo, $(1/20; 1/8; 1/5)$ é uma de tais
progressões, de tamanho $3$ e razão $3/40$. Mais ainda, esta é uma progressão
maximal em $S$ de tamanho $3$, pois ela não pode ser estendida à esquerda ou à
direita ($-1/40,11/40$ não são elementos de $S$). Mostre que para todo
$n \in \mathbb{N}$ existe uma progressão aritmética maximal de tamanho $n$ em
$S$ .\source{British Mathematical Olympiad 1997}

\textbf{M 9. } Suponha que
$$ \prod_{n=1}^{1996} (1 + nx^{3^{n}}) = 1 + a_{1} x^{k_{1}} + a_{2} x^{k_{2}} +
\cdots + a_{m} x^{k_{m}} $$ onde $a_{1}$, $a_{2}$,..., $a_{m}$ são não-nulos e
$k_{1}< k_{2} < \cdots < k_{m}$. Encontre $a_{1996}$. \source{Turkey 1996}

\textbf{M 10. } Seja $p$ um primo ímpar. Mostre que existe no máximo um
triângulo inteiro não-degenerado com perímetro $4p$ e área inteira. Caracterize
estes primos para os quais tal triângulo exista. \source{CRUX, Problem 2331,
  Paul Yiu}

\textbf{M 11. } Para cada inteiro positivo $n$, prove que existem dois inteiros
consecutivos, cada um dos quais é produto de $n$ inteiros positivos maiores que
$1$. \source{[Rh, pp. 165] Unused problems for 1985 CanMO}

\textbf{M 12. } Seja

$$
\begin{array}{cccc}
     a_{1,1} & a_{1,2} & a_{1,3} & \dots \\
     a_{2,1} & a_{2,2} & a_{2,3} & \dots \\
     a_{3,1} & a_{3,2} & a_{3,3} & \dots \\
     \vdots & \vdots & \vdots & \ddots
\end{array} $$

uma tabela dupla infinita de inteiros positivos, e suponha que cada inteiro
positivo aparece exatamente oito vezes na tabela. Prove que $a_{m,n} > mn$ para
algum par de inteiros positivos $(m,n)$. \source{Putnam 1985/B3}

\textbf{M 13. } A soma dos dígitos de um número natural $n$ é denotada por
$S(n)$. Prove que $S(8n) \ge \frac{1}{8} S(n)$ para cada $n$. \source{Latvia
  1995}

\textbf{M 14. } Seja $p$ um primo ímpar. Determine inteiros positivos $x$ e $y$
para os quais $x \le y$ e $\sqrt{2p}-\sqrt{x}-\sqrt{y}$ é não negativo e tão
pequeno quanto possível. \source{IMO Short List 1992 P17}

\textbf{M 15. } Seja $\alpha(n)$ o número de dígitos iguais a 1 na representação
diádica do inteiro positivo $n$. Prove que

\begin{enumerate} 
\item a desigualdade $\alpha(n^2 ) \le \frac{1}{2} \alpha(n) (1+\alpha(n))$ vale,
\item a igualdade é obtida para infinitos $n \in \mathbb{N}$, 
\item existe uma sequência $\{n_i\}$ tal que $\lim_{i \to \infty} \frac{
    \alpha({n_{i}}^2 )}{ \alpha(n_{i}) } = 0$.
\end{enumerate} \source{}

\textbf{M 16. } Mostre que se $a$ e $b$ são inteiros positivos, então
$$ \left( a+ \frac{1}{2} \right)^{n} + \left( b+ \frac{1}{2} \right)^{n} $$ é um
inteiro apenas para um número finito de inteiros positivos $n$. \source{[Ns
  pp.4]}

\textbf{M 17. } Determine o valor máximo de $m^{2}+n^{2}$, onde $m$ e $n$ são
inteiros satisfazendo $m,n\in \{1,2,...,1981\}$ e $(n^{2}-mn-m^{2})^{2}=1.$
\source{IMO 1981/3}

\textbf{M 18. } Denote por $S$ o conjunto de todos os primos $p$ tais que a
representação decimal de $\frac{1}{p}$ tem o período fundamental múltiplo de
$3$. Para cada $p \in S$ tal que $\frac{1}{p}$ tenha o período fundamental $3r$
podemos escrever
$$ \frac{1}{p} = 0.a_{1} a_{2} \cdots a_{3r} a_{1} a_{2} \cdots a_{3r}
\cdots, $$ onde $r=r(p)$. Para cada $p \in S$ e todo inteiro $k \ge 1$ defina
$$ f(k, p)=a_{k}+a_{k+r(p)}+a_{k+2r(p)}. $$

\begin{enumerate}
\item Prove que $S$ é finito.
\item Encontre o maior valor de $f(k, p)$ para $k \ge 1$ e $p \in S$.
\end{enumerate} \source{IMO Short List 1999 N4}

\textbf{M 19. } Determine todos os pares $(a, b)$ de números reais tais que
$a\lfloor bn\rfloor =b\lfloor an\rfloor$ para todos os inteiros positivos
$n$. \source{IMO Short List 1998 P15}

\textbf{M 20. } Seja $n$ um inteiro positivo que não é cubo perfeito. Defina os
números reais $a$, $b$, $c$ como
$$ a=\sqrt[3]{n}, \; b=\frac{1}{a-\lfloor a\rfloor}, \; c=\frac{1}{b-\lfloor
  b\rfloor}.$$ Prove que existem infinitos inteiros $n$ com a propriedade que
existem inteiros $r$, $s$, $t$, não todos nulos, tais que
$ra+sb+tc=0$. \source{IMO Short List 2002 A5}

\textbf{M 21. } Encontre, com prova, o número de inteiros positivos cuja
representação na base $n$ consiste de dígitos distintos com a propriedade de
que, exceto pelo dígito mais à esquerda, cada dígito difere de $\pm 1$ de algum
dígito mais à esquerda. \source{USA 1990}

\textbf{M 22. } A expansão decimal do número natural $a$ consiste de $n$
dígitos, enquanto a de $a^3$ consiste de $m$ dígitos. Pode $n+m$ ser igual a
$2001$? \source{[Tt] Tournament of the Towns 2001 Spring/O-Level}

\textbf{M 23. } Observe que
$$ \frac{1}{1}+\frac{1}{3}=\frac{4}{3}, \;\; 4^2 +3^2 =5^2, $$
$$ \frac{1}{3}+\frac{1}{5}=\frac{8}{15}, \;\; 8^2 +{15}^2 ={17}^2, $$
$$ \frac{1}{5}+\frac{1}{7}=\frac{12}{35}, \;\; {12}^2 +{35}^2 ={37}^2. $$
Estabeleça e prove uma generalização sugerida por estes exemplos. \source{[EbMk,
  pp. 10]}

\textbf{M 24. } Um número $n$ é chamado de número de Niven (em nome de Ivan
Niven) se ele é divisível pela soma de seus dígitos. Por exemplo, $24$ é um
número de Niven. Perove que não é possível existir mais de $20$ números de Niven
consecutivos. \source{(C. Cooper, R. E. Kennedy) [Jjt, pp. 58]}

\textbf{M 25. } Prove que não existe número natural que, após transferir seu
dígito inicial para o final, é aumentado seis, sete ou oito vezes. \source{[DNI, 12]}

\textbf{M 26. } Que inteiros têm a seguinte propriedade? Se o dígito final é
apagado, o inteiro é múltiplo do novo número. \source{[DNI, 11]}

\textbf{M 27. } Seja $A$ o conjunto dos dezesseis primeiros inteiros
positivos. Encontre o menor inteiro positivo $k$ satisfazendo a condição: Em
cada subconjunto de $k$ elementos de $A$, existem dois inteiros $a, b \in A$
tais que $a^2 + b^2$ é primo. \source{Vietnam 2004}

\textbf{M 28. } Qual é o dígito não-nulo mais à direita de $1000000!$?
\source{[JDS, pp. 28]}

\textbf{M 29. } Para quantos inteiros positivos $n$ o valor $$ \left(
  1999+\frac{1}{2}\right)^n + \left(2000+\frac{1}{2}\right)^n $$ é inteiro?
\source{[JDS, pp. 30]}

\textbf{M 30. } Um quadrado mágico é uma matriz $n\times n$ contendo os números
$1,2,\cdots n^2$ exatamente uma vez cada, tal que a soma dos elementos em cada
linha, coluna e diagonais principais é igual.

Existe algum quadrado mágico consistindo de números de Fibonacci distintos (dois
$1$'s são permitidos)? \source{[JDS, pp. 31]}

\textbf{M 31. } Alice e Bob jogam o seguinte jogo de adivinhação de
número. Alice escreve uma lista de inteiros positivos $x_{1}$, $\cdots$,
$x_{n}$, mas não os revela a Bobm que irá tentar determinar os números fazendo
perguntas a Alice. Bob escolhe uma lista de inteiros positivos $a_{1}$,
$\cdots$, $a_{n}$ e pede a Alice para contar o valor de
$a_{1}x_{1}+\cdots+a_{n}x_{n}$. Então Bob escolhe outra lista de inteiros
positivos $b_{1}$, $\cdots$, $b_{n}$ e pergunta a Alice
$b_{1}x_{1}+\cdots+b_{n}x_{n}$. O jogo continua até Bob ser capaz de determinar
os números de Alice. Quantas rodadas Bob precisa a fim de determinar os números
de Alice? \source{[JDS, pp. 57]}

\textbf{M 32. } Quatro números pares consecutivos são eliminados do conjunto $$
A=\{ 1, 2, 3, \cdots, n \}. $$ Se a média aritmética dos números restantes é
$51.5625$, quais foram os quatro números removidos? \source{[Rh2, pp. 78]}

\textbf{M 33. } Seja $S_{n}$ a soma dos dígitos de $2^n$. Prove ou disprove se
$S_{n+1}=S_{n}$ para algum inteiro positivo $n$. \source{MM Nov. 1982, Q679,
  M. S. Klamkin and M. R.Spiegel}

\textbf{M 34. } Contando da extremidade mais à direita, qual é o $2500$ - ésimo
dígito de $10000!$? \source{MM Sep. 1980, Problem 1075, Phillip M.Dunson}

\textbf{M 35. } Para cada número natural $n$, denote $Q(n)$ a soma dos dígitos
da representação decimal de $n$. Prove que existem infinitos números naturais
$k$ com $Q(3^{k})>Q(3^{k+1})$. \source{Germany 1996}

\textbf{M 36. } Sejam $n$ e $k$ inteiros com $n>0$. Prove que
$$  -\frac{1}{2n} \sum^{n-1}_{m=1} \cot \frac{\pi m}{n} \sin \frac{2\pi km}{n}
= \begin{cases}
  \tfrac{k}{n}-\lfloor\tfrac{k}{n}\rfloor-\frac12 & \text{ se }k|n \\
  0 & \text{ caso contrário }
\end{cases}. $$ \source{[Tma, pp.175]}

\textbf{M 37. } A função $\mu: \mathbb{N} \to \mathbb{C}$ é definida por
$$ \mu(n) = \sum_{k \in R_{n}} \left( \cos \frac{2k\pi}{n} + i \sin
  \frac{2k\pi}{n} \right), $$ onde
$R_{n}=\{ k \in \mathbb{N} \vert 1 \le k \le n, \gcd(k, n)=1 \}$. Mostre que
$\mu(n)$ é um inteiro para todos os inteiros positivos $n$. \source{}

\textbf{M 38. } Considere $n$ inteiros positivos $a_{1}, \ldots,
a_{n}$. Quaisquer dois números $a_{i}$ e $a_{k}$ com $i < k$ podem ser
substituídos como se segue: $a_{i}$ é substituído por $\gcd(a_{i},a_{k})$ e
$a_{k}$ é substituído por $\text{lcm}(a_{i},a_{k})$. Prove que a sequência é
modificada apenas um número finito de vezes. \source{}

%%% Local Variables:
%%% mode: latex
%%% coding: utf-8-unix
%%% fill-column: 80
%%% TeX-master: "MASTER"
%%% End:
