\chapter{Números Racionais}

\quoting{Deus fez os inteiros, todo o resto é trabalho do homem.}{Leopold
  Kronecker }

\textbf{D 1. } Suppose that a rectangle with sides $a$ and $b$ is arbitrarily cut into $n$ squares with sides $x_{1},\ldots,x_n$. Show that $\frac{x_{i}}{a} \in \mathbb{Q}$  and $\frac{x_{i}}{b} \in \mathbb{Q}$ for all $i \in \{1, \cdots, n \}$. \source{[Vvp, pp. 40]}

\textbf{D 2. } Find all $x$ and $y$ which are rational multiples of $\pi$ with $0<x<y<\frac{\pi}{2}$ and $\tan x+\tan y =2$. \source{CRUX, Problem 1632, Stanley Rabinowitz}

\textbf{D 3. } Let $\alpha$ be a rational number with $0<\alpha<1$ and $\cos (3 \pi \alpha)+2\cos(2 \pi \alpha)=0$. Prove that $\alpha=\frac{2}{3}$. \source{IMO ShortList 1991 P19 (IRE 5)}

\textbf{D 4. } Suppose that $\tan \alpha =\frac{p}{q}$, where $p$ and $q$ are integers and $q \neq 0$. Prove the number $\tan \beta$ for which $\tan 2\beta =\tan 3\alpha$ is rational only when $p^2 +q^2$ is the square of an integer. \source{IMO Long List 1967 P20 (DDR)}

\textbf{D 5. } Prove that there is no positive rational number $x$ such that $$ x^{\lfloor x\rfloor  }=\frac{9}{2}. $$ \source{Austria 2002}

\textbf{D 6. } Let $x, y, z$ non-zero real numbers such that $xy$, $yz$, $zx$ are rational. \begin{enumerate} \item Show that the number $x^{2}+y^{2}+z^{2}$ is rational. \item If the number $x^{3}+y^{3}+z^{3}$ is also rational, show that $x$, $y$, $z$ are rational. \end{enumerate} \source{Romania 2001, Marius Ghergu}

\textbf{D 7. } If $x$ is a positive rational number, show that $x$ can be uniquely expressed in the form $$ x=a_{1}+\frac{a_2}{2!}+\frac{a_3}{3!}+\cdots, $$ where $a_1a_2,\cdots$ are integers, $0 \le a_n \le n-1$ for $n>1$, and the series terminates. Show also that $x$ can be expressed as the sum of reciprocals of different integers, each of which is greater than $10^6$. \source{IMO Long List 1967 (GB)}

\textbf{D 8. } Find all polynomials $W$ with real coefficients possessing the following property: if $x+y$ is a rational number, then $W(x)+W(y)$ is rational. \source{Poland 2002}

\textbf{D 9. } Prove that every positive rational number can be represented in the form $$ \frac{a^3 + b^3}{c^3 + d^3} $$ for some positive integers $a, b, c$, and $d$. \source{IMO Short List 1999}

\textbf{D 10. } The set $S$ is a finite subset of $[0,1]$ with the following property: for all $s \in S$, there exist $a,b\in S \cup \{0,1\}$ with $a, b \neq s$ such that $s=\frac{a+b}{2}$. Prove that all the numbers in $S$ are rational. \source{Berkeley Math Circle Monthly Contest 1999-2000}

\textbf{D 11. } Let $S=\{x_0, x_1, \cdots, x_n\} \subset [0,1]$ be a finite set of real numbers with $x_{0}=0$ and $x_{1}=1$, such that every distance between pairs of elements occurs at least twice, except for the distance $1$. Prove that all of the $x_i$ are rational. \source{Iran 1998}

\textbf{D 12. } Does there exist a circle and an infinite set of points on it such that the distance between any two points of the set is rational? \source{[Zh, PP. 40] Mediterranean MC 1999 (Proposed by Ukraine)}

\textbf{D 13. } Let $k$ and $m$ be positive integers. Show that $$ S(m, k)=\sum_{n=1}^{\infty} \frac{1}{n(mn+k)} $$ is rational if and only if $m$ divides $k$. \source{[PbAw, pp. 2]}

\textbf{D 14. } Prove that for any distinct rational numbers $a, b, c$, the number $$ \frac{1}{(b-c)^2}+ \frac{1}{(c-a)^2}+ \frac{1}{(a-b)^2} $$ is the square of some rational number. \source{[EbMk, pp. 23]}

\textbf{D 15. } Let $a_i,b_i\in\mathbb{Q}$ $(i=1,2, \cdots, n)$ such that $\forall x\in\mathbb{R}$ we have that $$x^2+x+4= \sum_{i=1}^{n} (a_ix+b)^2.$$ Find the least possible value of n. \source{China TST 2006}

\textbf{D 16. } Prove that for any positive integers $a$ and $b$ $$ \left\vert a\sqrt{2} -b \right\vert > \frac{1}{2(a+b)}. $$ \source{(A. Mirotin) Belarus 2002}

\textbf{D 17. } Prove that there exist positive integers $m$ and $n$ such that $$ \left\vert \frac{m^2}{n^3} - \sqrt{2001} \right\vert < \frac{1}{10^{8}}. $$ \source{(V. Bernik) Belarus 2001}

\textbf{D 18. } Let $a, b, c$ be integers, not all zero and each of absolute value less than one million. Prove that $$ \left\vert a+b\sqrt{2}+c\sqrt{3} \right\vert > \frac{1}{10^{21}}. $$ \source{Putnam 1980}

\textbf{D 19. } Let $a, b, c$ be integers, not all equal to $0$. Show that $$ \frac{1}{4a^2 +3b^2 +2c^2} \le \vert \sqrt[3]{4} a + \sqrt[3]{2} b +c \vert. $$ \source{CRUX(No. 7, Volume 25, 1999), Problem A240, Mohammed Aassila}

\textbf{D 20. } Prove that for any irrational number $\xi$, there are infinitely many rational numbers $\frac{m}{n}$ $\left( (m,n) \in \mathbb{Z} \times \mathbb{N} \right)$ such that $$ \left\vert \xi - \frac{n}{m} \right\vert < \frac{1}{\sqrt{5} m^2}. $$ \source{}

\textbf{D 21. } Show that $e=\sum^{\infty}_{n=0} \frac{1}{n!}$ is irrational. \source{}

\textbf{D 22. } Show that $\cos \frac{\pi}{7}$ is irrational. \source{}

\textbf{D 23. } Show that $\frac{1}{\pi} \arccos \left( \frac{1}{\sqrt{2003}} \right)$ is irrational. \source{}

\textbf{D 24. } Show that $\cos 1^{\circ}$ is irrational. \source{}

\textbf{D 25. } An integer-sided triangle has angles $p \theta$ and $q \theta$, where $p$ and $q$ are relatively prime integers. Prove that $\cos \theta$ is irrational. \source{CRUX(No. 1. Vol. 24, 1998), Problem 2305, Richard I. Hess}

\textbf{D 26. } It is possible to show that $\csc \frac{3 \pi}{29} -\csc \frac{10 \pi}{29}=1.999989433...$. Prove that there are no integers $j$, $k$, $n$ with odd $n$ satisfying $\csc \frac{j \pi}{n} -\csc \frac{k \pi}{n}=2$. \source{AMM, Problem 10630, Richard Strong}

\textbf{D 27. } For which angles $\theta$, with $\theta$ a rational number of degrees, is ${\tan}^2 \theta + {\tan}^2 2\theta $ is irrational? \source{}

\textbf{D 28. } Show that the cube roots of three distinct primes cannot be terms in an arithmetic progression. \source{[KhKw, pp. 11]}

\textbf{D 29. } Let $n$ be an integer greater than or equal to $3$.  Prove that there is a set of $n$ points in the plane such that the distance between any two points is irrational and each set of three points determines a non-degenerate triangle with a rational area. \source{[AI, pp. 116] For a proof, See [KaMr].}

\textbf{D 30. } You are given three lists A, B, and C.  List A contains the numbers of the form $10^k$ in base 10, with $k$ any integer greater than or equal to 1.  Lists B and C contain the same numbers translated into base 2 and 5 respectively:  $$\begin{array}{lll} A & B & C \\ 10 & 1010 & 20 \\ 100 & 1100100 & 400 \\ 1000 & 1111101000 & 13000 \\ \vdots & \vdots & \vdots \end{array}.$$ Prove that for every integer $n > 1$, there is exactly one number in exactly one of the lists B or C that has exactly $n$ digits. \source{APMO 1994/5}

\textbf{D 31. } Prove that if $\alpha$ and $\beta$ are positive irrational numbers satisfying $\frac{1}{\alpha}+\frac{1}{\beta}=1$, then the sequences $$ \lfloor\alpha\rfloor, \lfloor 2\alpha\rfloor, \lfloor 3\alpha\rfloor, \cdots $$ and $$ \lfloor\beta\rfloor, \lfloor 2\beta\rfloor, \lfloor 3\beta\rfloor, \cdots $$ together include every positive integer exactly once.    \source{}

\textbf{D 32. } For a positive real number $\alpha$, define $$ S(\alpha)=\{ \lfloor n\alpha\rfloor   \; \vert \; n=1,2,3,\cdots \}. $$ Prove that $\mathbb{N}$ cannot be expressed as the disjoint union of three sets $S(\alpha)$, $S(\beta)$, and $S(\gamma)$. \source{Putnam 1995/B6}

\textbf{D 33. } Let $f(x)=\prod_{n=1}^{\infty} \left( 1 + \frac{x}{2^n} \right)$. Show that at the point $x=1$, $f(x)$ and all its derivatives are irrational. \source{}

\textbf{D 34. } Let $\{a_n\}_{n \ge 1}$ be a sequence of positive numbers such that $$ a_{n+1}^2 = a_{n}+1, \;\; n \in \mathbb{N}. $$ Show that the sequence contains an irrational number. \source{}

\textbf{D 35. } Show that $\tan \left(  \frac{\pi}{m} \right)$ is irrational for all positive integers $m \ge 5$. \source{MM, Problem 1385, Howard Morris}

\textbf{D 36. } Prove that if $g \ge 2$ is an integer, then two series $$ \sum_{n=0}^{\infty} \frac{1}{g^{n^2}} \;\; \text{and} \;\; \sum_{n=0}^{\infty} \frac{1}{g^{n!}} $$ both converge to irrational numbers. \source{[Ae, pp. 226]}

\textbf{D 37. } Seja $1<a_{1}<a_{2}<\cdots$ uma sequência de inteiros
positivos. Mostre que $$ \frac{2^{a_1}}{{a_{1}}!} +\frac{2^{a_2}}{{a_{2}}!}
+\frac{2^{a_3}}{{a_{3}}!} + \cdots $$ é irracional. \source{[PeJs, pp. 95]}

\textbf{D 38. } Existem números reais $a$ e $b$ tais que
\begin{enumerate}
\item $a+b$ é racional e $a^n +b^n $ é irracional para todo $n \in \mathbb{N}$
  com $n \ge 2$?
\item $a+b$ é irracional e $a^n +b^n $ é racional para todo $n \in \mathbb{N}$
  com $n \ge 2$?
\end{enumerate} \source{(N. Agahanov) [PeJs, pp. 99]}

\textbf{D 39. } Seja $p(x)=x^{3}+a_{1}x^{2}+a_{2}x+a_{3}$ com coeficientes reais
e raízes $r_{1}$, $r_{2}$, e $r_{3}$. Se $r_{1}-r_{2}$ é racional, devem
$r_{1}$, $r_{2}$, and $r_{3}$ ser racionais? \source{[PbAw, pp. 2]}

\textbf{D 40. } Seja $\alpha=0.d_{1}d_{2}d_{3} \cdots$ a representação decimal
de um número real entre $0$ e $1$. Seja $r$ um número real com $\vert r
\vert<1$.

\begin{enumerate}
\item Se $\alpha$ e $r$ são racionais, deve $\sum_{i=1}^{\infty} d_{i}r^{i}$ ser
  racional?
\item Se $\sum_{i=1}^{\infty} d_{i}r^{i}$ e $r$ são racionais, $\alpha$ deve ser
  racional?
\end{enumerate} \source{[Ams, pp. 14]}

%%% Local Variables:
%%% mode: latex
%%% coding: utf-8-unix
%%% fill-column: 80
%%% TeX-master: "MASTER"
%%% End:
