\chapter{Teoria Geométrica dos Números}

\textbf{L 1. } Existe um pentágono convexo, com todos os seus vértices em pontos
do reticulado, sem pontos do reticulado em seu interior?
\source{J. R. Arkinstall, Bull. of the Australian Math. Soc., 1980}

\textbf{L 2. } Mostre que não existem quatro pontos no plano euclidiano tal que
as distâncias dois a dois entre estes pontos sejam todas inteiros ímpares.
\source{Putnam 1993/B5}

\textbf{L 3. } Prove que três pontos do reticulado plano jamais formam um
triângulo equilátero. \source{}

\textbf{L 4. } Os comprimentos dos lados de um polígono com $1994$ lados são
$a_{i}=\sqrt{i^2 +4}$ $ \; (i=1,2,\cdots,1994)$. Prove que seus vértices não
podem ser todos eles pontos do reticulado. \source{Israel 1994}

\textbf{L 5. } Um triângulo tem pontos do reticulado como vértices e não contém
nenhum outro pontos do reticulado. Prove que sua área éa $\frac{1}{2}$. \source{}

\textbf{L 6. } Seja $R$ uma região simétrica convexa ao longo da origem com área
maior que 4. Mostre que $R$ deve conter um ponto do reticulado diferente da
origem. \source{[Hua pp.535]}

\textbf{L 7. } Mostre que o número $r(n)$ de representações de $n$ como soma de
dois quadrados tem $\pi$ como média aritmética, isto é,
$$ \lim_{n \to \infty} \frac{1}{n} \sum^{n}_{m=1} r(m) = \pi. $$ \source{[GjJj
  pp.215]}

\textbf{L 8. } Prove que em um plano coordenado é impossível desenhar uma linha
quebrada tal que

\begin{itemize}
\item as coordenadas de cada vértice são racionais,
\item o comprimento de cada aresta é igual a $1$,
\item a linha tem um número ímpar de vértices.
\end{itemize} \source{IMO Short List 1990 USS3}

\textbf{L 9. } Prove que se um triângulo do reticulado não tem pontos em sua
fronteira além dos vértices, e um ponto em seu interior, então este ponto é seu
centro de gravidade. \source{[PeJs, pp. 125]}

\textbf{L 10. } Encontre coordenadas de um conjunto de oito pontos não
colineares tais que cada um está a uma distância inteira dos
outros. \source{[Jjt, pp. 75]}

%%% Local Variables:
%%% mode: latex
%%% coding: utf-8-unix
%%% fill-column: 80
%%% TeX-master: "MASTER"
%%% End:
