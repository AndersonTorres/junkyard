\chapter{Equações Diofantinas}

\quoting{Dividir um cubo em dois outros cubos, uma quarta potência ou em geral
  qualquer potência que seja em duas potências de mesma denominação acima da
  segunda é impossível, e eu certamente encontrei uma prova admirável disso, mas
  a margem é estreita demais para contê-la.}{Pierre de Fermat, na margem de sua
  cópia de Arithmetica de Diofanto}

\textbf{E 1. } Existe um inteiro positivo $n$ tal que $$ n^{5} = 133^{5} +
110^{5} + 84^{5} + 27^{5}. $$ Encontre o valor de $n$. \source{AIME 1989/9}

\textbf{E 2. } O número $21982145917308330487013369$ é a décima-terceira
potência de um inteiro positivo. Qual é este inteiro positivo? \source{UC
  Berkeley Preliminary Exam 1983}

\textbf{E 3. } Existe solução para a equação
$$ x^{2}+y^{2}+z^{2}+u^{2}+v^{2}=xyzuv-65 $$ em inteiros com $x, y, z, u, v$
maiores que $1998$? \source{Taiwan 1998}

\textbf{E 4. } Encontre todos os pares $(x, y)$ de racionais positivos tais que
$x^{2}+3y^{2}=1$. \source{}

\textbf{E 5. } Encontre todos os pares $(x, y)$ de racionais tais que
$y^2 =x^3 -3x+2$. \source{}

\textbf{E 6. } Mostre que existem infinitos pares $(x, y)$ de números racionais
tais que $x^3 +y^3 =9$. \source{}

\textbf{E 7. } Determine todos os pares $(x,y)$ de inteiros positivos
satisfazendo a equação $$ (x+y)^{2}-2(xy)^{2}=1. $$ \source{Poland 2002}

\textbf{E 8. } Mostre que a equação $$ x^{3}+y^{3}+z^{3}+t^{3}=1999 $$ tem
infinitas soluções inteiras. \source{Bulgaria 1999}

\textbf{E 9. } Determine todos os inteiros $a$ para os quais a equação $$
x^{2}+axy+y^{2}=1 $$ tenha infinitas soluções inteiras distintas $x,
\;y$. \source{Ireland 1995}

\textbf{E 10. } Prove que existem inteiros positivos únicos $a$ e $n$ tais
que $$ a^{n+1}-(a+1)^{n}= 2001. $$ \source{Putnam 2001}

\textbf{E 11. } Encontre todos $(x,y,n) \in {\mathbb{N}}^3$ tais que
$\gcd(x, n+1)=1$ e $x^{n}+1=y^{n+1}$. \source{India 1998}

\textbf{E 12. } Encontre todos $(x,y,z) \in {\mathbb{N}}^3$ tais que
$x^{4}-y^{4}=z^{2}$. \source{}

\textbf{E 13. } Encontre todos os pares $(x,y)$ de inteiros positivos que
satisfazem a equação $$ y^{2}=x^{3}+16. $$ \source{Italy 1994}

\textbf{E 14. } Mostre que a equação $x^2 +y^5 =z^3$ tem infinitas soluções em
inteiros $x, y, z$ para os quais $xyz \neq 0$. \source{Canada 1991}

\textbf{E 15. } Prove que não existem inteiros $x$ e $y$ satisfazendo
$x^{2}=y^{5}-4$. \source{Balkan Mathematical Olympiad 1998}

\textbf{E 16. } Encontre todos os pares $(a,b)$ de inteiros positivos diferentes
que satisfazem $W(a)=W(b)$, em que $W(x)=x^{4}-3x^{3}+5x^{2}-9x$. \source{}

\textbf{E 17. } Encontre todos os inteiros positivos $n$ para os quais a
equação $$a+b+c+d=n \sqrt{abcd}$$ tem uma solução em inteiros positivos.
\source{Vietnam 2002}

\textbf{E 18. } Determine todas as soluções inteiras positivas $(x, y, z, t)$ da
equação $$ (x+y)(y+z)(z+x)=xyzt $$ para as quais
$\gcd(x, y)=\gcd(y, z)=\gcd(z, x)=1$. \source{Romania 1995, M. Becheanu}

\textbf{E 19. } Encontre todos $(x, y, z, n) \in {\mathbb{N}}^4$ tais que
$ x^3 +y^3 +z^3 =nx^2 y^2 z^2$. \source{[UmDz pp.14] Unused Problem for the
  Balkan MO}

\textbf{E 20. } Determine todos os inteiros positivos $n$ para os quais a
equação $$x^n + (2+x)^n + (2-x)^n = 0$$ tenha um inteiro como solução.
\source{APMO 1993/4}

\textbf{E 21. } Prove que a equação $$6(6a^2 + 3b^2 + c^2) = 5n^2$$ não tem
solução em inteiros exceto $a=b=c=n=0$. \source{APMO 1989/2}

\textbf{E 22. } Encontre todos os inteiros $a,b,c,x,y,z$ tais que
$$ a+b+c=xyz, \; x+y+z=abc, \; a \ge b \ge c \ge 1, \; x \ge y \ge z \ge 1. $$
\source{Poland 1998}

\textbf{E 23. } Encontre todos os $(x,y,z) \in {\mathbb{Z}}^3$ tais que
$x^{3}+y^{3}+z^{3}=x+y+z=3$. \source{}

\textbf{E 24. } Prove que se $n$ é um inteiro positivo tal que a equação $$
x^{3}-3xy^{2}+y^{3}=n. $$ tem uma solução em inteiros $(x,y),$ então ela tem
pelo menos três soluções em inteiros. Mostre que a equação não tem solução em
inteiros quando $n=2891$. \source{IMO 1982/4}

\textbf{E 25. } Qual é o menor inteiro positivo $t$ tal que existam inteiros
$x_{1},x_{2}, \cdots, x_{t}$ com
$$ {x_{1}}^{3}+{x_{2}}^{3}+\cdots+{x_{t}}^{3}=2002^{2002} \;\;? $$
\source{IMO Short List 2002 N1}

\textbf{E 26. } Resolva em inteiros a seguinte equação
$$ n^{2002}=m(m+n)(m+2n)\cdots(m+2001n). $$ \source{Ukraine 2002}

\textbf{E 27. } Prove que existem infinitos inteiros positivos $n$ tais que
$p=nr$, onde $p$ e $r$ são respectivamente o semiperímetro e o inraio de um
triângulo de lados com medidas inteiras. \source{IMO Short List 2000 N5}

\textbf{E 28. } Sejam $a, b, c$ inteiros positivos tais que $a$ e $b$ são primos
entre si e $c$ é primo com $a$ ou $b$. Prove que existem infinitas triplas
$(x, y, z)$ de inteiros positivos distintos tais que $$ x^{a} +y^{b}= z^{c}. $$
\source{IMO Short List 1997 N6}

\textbf{E 29. } Encontre todos os pares de inteiros $(x, y)$ satisfazendo a
igualdade $$ y(x^2 +36)+x(y^2 -36)+y^2 (y-12)=0.$$ \source{Belarus 2000}

\textbf{E 30. } Seja, $a$, $b$, $c$ inteiros dados, $a\ge 0$, $ac-b^2=p$ um
inteiro positivo livre de quadrados. Seja $M(n)$ o número de pares de inteiros
$(x, y)$ para os quais $ax^2 +bxy+cy^2=n$. Prove que $M(n)$ é finito e
$M(n)=M(p^{k} \cdot n)$ para todo inteiro $k \ge 0$. \source{IMO Short List 1993
  G3}

\textbf{E 31. } Determine todas as soluções inteiras do sistema
$$ 2uv-xy=16, $$ $$ xv-yu=12. $$
\source{[Eb1, pp. 19] AMM 61(1954), 126; 62(1955), 263}

\textbf{E 32. } Seja $n$ um número natural. Resolva nos números inteiros a
equação
$$ x^n +y^n =(x-y)^{n+1}. $$ \source{IMO Long List 1987 (Romania)}

\textbf{E 33. } Existe um inteiro positivo tal que seu cubo é igual a
$3n^2 +3n+7$, onde $n$ é inteiro? \source{IMO Long List 1967 P (PL)}

\textbf{E 34. } Existem inteiros $m$ e $n$ tais que $5m^2 -6mn+7n^2 =1985$?
\source{IMO Long List 1985 (SE1)}

\textbf{E 35. } Encontre todos os polinômios cúbicos $x^3 +ax^2 +bx+c$ admitindo
os números racionais $a$, $b$ e $c$ como raízes. \source{IMO Long List 1985 (TR3)}

\textbf{E 36. } Prove que a equação $a^2 +b^2 =c^2 +3$ tem infinitas soluções
inteiras $(a, b, c)$. \source{Italy 1996}

\textbf{E 37. } Prove que para cada inteiro positivo $n>2$ existem inteiros
positivos ímpares $x_n$ e $y_n$ tais que ${x_{n}}^2 +7{y_{n}}^2
=2^n$. \source{Bulgaria 1996}

\textbf{E 38. } Suponha que $p$ é um primo ímpar tal que $2p+1$ é primo
também. Mostre que a equação $x^{p}+2y^{p}+5z^{p}=0$ não tem solução em inteiros
além de $(0,0,0)$. \source{[JeMm, pp. 10]}

\textbf{E 39. } Sejam $A, B, C, D, E$ inteiros, $B \neq 0$ e
$F=AD^{2}-BCD+B^{2}E \neq 0$. Prove que o número $N$ de pares de inteiros
$(x, y)$ tais que $$ Ax^2 +Bxy+Cx+Dy+E=0, $$ satisfaz
$N \le 2 d( \vert F \vert )$, onde $d(n)$ denota o número de divisores positivos
do inteiro positivo $n$.  \source{[KhKw, pp. 9]}

\textbf{E 40. } Determine todos os pares de números racionais $(x, y)$ tais que
$$ x^3 +y^3 = x^2 +y^2. $$ \source{[EbMk, pp. 44]}

\textbf{E 41. } Encontre todos os inteiros $a$ para os quais $x^3 -x+a$ tem três
raízes inteiras. \source{[GML, pp. 2]}

\textbf{E 42. } Encontre todas as soluções inteiras de
$x^{3}+2y^{3}=4z^{3}$. \source{[GML, pp. 33]}

\textbf{E 43. } Para todos $n \in \mathbb{N}$, mostre que o número de soluções
inteiras $(x, y)$ de $$ x^{2}+xy+y^{2}=n $$ é finito e múltiplo de
$6$. \source{[GML, pp. 192]}

\textbf{E 44. } Mostre que não pode haver quatro quadrados em uma progressão
aritmética. \source{(Fermat) [Ljm, pp. 21]}

\textbf{E 45. } Sejam $a, b, c, d, e, f$ inteiros tais que $b^2 -4ac>0$ não é
quadrado perfeito e $4acf+bde-ae^2 -cd^2 -fb^2 \neq 0$. Seja
$$ f(x, y)=ax^2 +bxy +cy^2 +dx+ey+f $$
Suponha que $f(x, y)=0$ tenha uma solução inteira. Mostre que $f(x, y)=0$ tem
infinitas soluções inteiras. \source{(Gauss) [Ljm, pp. 57]}

\textbf{E 46. } Mostre que a equação $x^4 +y^4 +4z^4 =1$ tem infinitas soluções
racionais. \source{[Ljm, pp. 94]}

\textbf{E 47. } Resolva a equação $x^2 +7=2^n$ em inteiros. \source{[Ljm,
  pp. 205]}

\textbf{E 48. } Mostre que as únicas soluções da equação $x^{3}-3xy^2 -y^3 =1$
são dadas por $(x,y)=(1,0),(0,-1),(-1,1),(1,-3),(-3,2),(2,1)$. \source{[Ljm,
  pp. 208]}

\textbf{E 49. } Mostre que a equação $y^{2}=x^{3}+2a^{3}-3b^2$ não tem soluções
em inteiros se $ab \neq 0$, $a \not\equiv 1 \; \pmod{3}$, $3$ não é divisor de
$b$, $a$ é ímpar se $b$ é par, e $p=t^2 +27u^2$ tem solução en inteiros
positivos $t,u$ se $p \vert a$ e $p \equiv 1 \; \pmod{3}$. \source{[Her,
  pp. 287]}

\textbf{E 50. } Prove que o produto de cinco inteiros positivos consecutivos
nunca é quadrado perfeito. \source{[Rh3, pp. 207]}

\textbf{E 51. } Existem dois triângulos retângulos com lados de medidas inteiras
que tenham os comprimentos de exatamente dois lados em comum? \source{}

\textbf{E 52. } Mostre que o número de triângulos retângulos de lados inteiros
cuja razão entre área e semi-perímetro é $p^{m}$, onde $p$ é primo e $m$ é
inteiro, é $m+1$ se $p=2$ e $2m+1$ se $p \neq 2$. \source{MM, Sep. 1980, Problem
  1077, Henry Klostergaard}

\textbf{E 53. } Dado que
$$ 34! = 295232799cd96041408476186096435ab000000_{(10)}, $$ determine os dígitos
$a, b, c$, e $d$. \source{British Mathematical Olympiad 2002/2003, 1-1}

\textbf{E 54. } Prove que a equação
$$\prod_{cyc}^7 (x_1-x_2)= \prod_{cyc}^7 (x_1-x_3)$$ tem uma solução em números
naturais onde todos os $x_i$ são diferentes. \source{Latvia 1995}

\textbf{E 55. } Mostre que a equação $\binom{n}{k}=m^{l}$ não tem solução
inteira com $l \ge 2$ e $4 \le k \le n-4$. \source{(P. Erd\"os) [MaGz pp.13-16]}

\textbf{E 56. } Resolva em inteiros positivos a equação
$10^{a}+2^{b}-3^{c}=1997$. \source{Belarus 1999, S. Shikh}

\textbf{E 57. } Resolva a equação $28^x =19^y +87^z$, onde $x, y, z$ são
inteiros. \source{IMO Long List 1987 (Greece)}

\textbf{E 58. } Mostre que a equação $x^7 + y^7 = {1998}^z$ não tem solução em
inteiros positivos. \source{[VsAs]}

\textbf{E 59. } Resolva a equação $2^x -5 =11^{y}$ em inteiros positivos.
\source{CRUX, Problem 1797, Marcin E. Kuczma}

\textbf{E 60. } Resolva a equação $7^x -3^y =4$ em inteiros positivos.
\source{India 1995}

\textbf{E 61. } Mostre que $\vert 12^m -5^n\vert \ge 7$ para todo
$m, n \in \mathbb{N}$. \source{}

\textbf{E 62. } Mostre que não existe inteiro positivo $k$ para o qual a
equação $$ (n-1)!+1=n^{k} $$ é verdadeira quando $n$ é maior que $5$.
\source{[Rdc pp.51]}

\textbf{E 63. } Determine todos os pares $(a, b)$ de inteiros tais que
$$ (19a+b)^{18}+(a+b)^{18}+(19b+a)^{18} $$ é um quadrado perfeito não-nulo.
\source{Austria 2002}

\textbf{E 64. } Seja $b$ um inteiro positivo. Determine todas as $2002$-tuplas
de inteiros não negativos $(a_{1}, a_{2}, \cdots, a_{2002})$ satisfazendo
$$ \sum^{2002}_{j=1} {a_{j}}^{a_{j}}=2002{b}^{b}. $$ \source{Austria 2002}

\textbf{E 65. } Existe um inteiro positivo $m$ tal que a equação
$$ \frac{1}{a}+ \frac{1}{b}+ \frac{1}{c}+ \frac{1}{abc} = \frac{m}{a+b+c} $$
tenha infinitas soluções em inteiros positivos $a, b, c \; $? \source{IMO Short
  List 2002 N4}

\textbf{E 66. } Considere o sistema $$ x+y=z+u, $$ $$ 2xy=zu. $$ Encontre o
maior valor da constante real $m$ tal que $m \le \frac{x}{y}$ para qualquer
solução inteira positiva $(x, y, z, u)$ do sistema, com $x \ge y$. \source{IMO
  Short List 2001 N2}

\textbf{E 67. } Determine todos os números racionais positivos  $r \neq 1$ tais
que $\sqrt[r-1]{r}$ é racional. \source{Hong Kong 2000}

\textbf{E 68. } Mostre que a equação $\{x^3\}+\{y^3\}=\{z^3\}$ tem infinitas
soluções racionais não-inteiras. \source{Belarus 1999}

\textbf{E 69. } Seja $n$ um inteiro positivo. Prove que a equação
$$ x+y+\frac{1}{x}+\frac{1}{y}=3n $$ não tem solução em números racionais
positivos. \source{Baltic Way 2002}

\textbf{E 70. } Encontre todos os pares $(x, y)$ de números racionais positivos
tais que $x^{y}=y^{x}$. \source{}

\textbf{E 71. } Encontre todos os pares $(a,b)$ de inteiros positivos que
satisfazem a equação $$ a^{b^2} = b^a.$$ \source{IMO 1997/5}

\textbf{E 72. } Encontre todos os pares $(a,b)$ de inteiros positivos que
satisfazem a equação $$ a^{a^a} = b^b.$$ \source{Belarus 2000}

\textbf{E 73. } Sejam $a,b$, e $x$ inteiros positivos tais que
$x^{a+b}=a^b{b}$. Prove que $a=x$ and $b=x^{x}$. \source{Iran 1998}

\textbf{E 74. } Encontre todos os pares $(m,n)$ de inteiros que satisfazem a
equação $$ (m-n)^{2}=\frac{4mn}{m+n-1}. $$ \source{Belarus 1996}

\textbf{E 75. } Encontre todos os ternos $(l, m, n)$ de inteiros positivos
primos entre si dois a dois tais que
$$ (l+m+n)\left( \frac{1}{l}+\frac{1}{m}+\frac{1}{n} \right) $$ é inteiro.
\source{Korea 1998}

\textbf{E 76. } Sejam $x, y$, e $z$ inteiros com $z>1$. Mostre que
$$ (x+1)^2 +(x+2)^2 + \cdots +(x+99)^2 \neq y^z. $$ \source{Hungary 1998}

\textbf{E 77. } Encontre todos os inteiros positivos $m$ and $n$ para os
quais $$ 1!+2!+3!+\cdots+n!=m^2.$$ \source{[Eb2, pp. 20] Q657, MM 52(1979), 47,
  55}

\textbf{E 78. } Prove que se $a, b, c, d$ são inteiros tais que
$d=( a+\sqrt[3]{2}b+\sqrt[3]{4}c)^{2}$ então $d$ é um quadrado
perfeito. \source{IMO Short List 1980 (GB)}

\textbf{E 79. } Encontre um par de números naturais de quatro dígitos $A$ e $B$
primos entre si tal que para todos os naturais $m$ e $n$,
$\vert A^m -B^n \vert \ge 400$. \source{[DfAk, pp. 18] Leningrad Mathematical
  Olympiad 1988}

\textbf{E 80. } Encontre todas as triplas $(a, b, c)$ de inteiros positivos para
a equação $$ a! b! = a! +b! +c!. $$ \source{British Mathematical Olympiad
  2002/2003, 1-5}

\textbf{E 81. } Encontre todos os pares $(a, b)$ de inteiros positivos tais que
$$ (\sqrt[3]{a}+\sqrt[3]{b}-1 )^2 = 49+20 \sqrt[3]{6}. $$ \source{British
  Mathematical Olympiad 2000, 2-3}

\textbf{E 82. } Para quais números positivos $a$ é
$$ \sqrt[3]{2+\sqrt{a}}+\sqrt[3]{2-\sqrt{a}} $$ um inteiro? \source{MM, Problem
  1529, David C. Kay}

\textbf{E 83. } Encontre todas as soluções inteiras de
$2(x^5 +y^5 +1)=5xy(x^2 +y^2 +1)$. \source{MM, Problem 1538, Murray S. Klamkin
  and George T. Gilbert.}

\textbf{E 84. } Um triângulo de lados inteiros é chamado heroniano se sua área é
inteira. Existe um triângulo heroniano cujos lados sejam média aritmética, média
geométrica e média harmônica de dois inteiros positivos? \source{CRUX, Problem
  2351, Paul Yiu}

\textbf{E 85. } Qual é o menor quadrado perfeito que termina em $9009$?
\source{[EbMk, pp. 22]}

\textbf{E 86. } {\it (Leo Moser)} Mostre que a equação diofantina
$$ \frac{1}{x_{1}}+ \frac{1}{x_{2}}+ \cdots +\frac{1}{x_{n}}+ \frac{1}{x_{1}
  x_{2} \cdots x_{n}} = 1 $$ tem pelo menos uma solução para cada inteiro
positivo $n$. \source{[EbMk, pp. 46]}

\textbf{E 87. } Prove que o número $99999+111111\sqrt{3}$ não pode ser escrito
na forma $(A+B\sqrt{3})^2$, onde $A$ e $B$ são inteiros. \source{[DNI, 42]}

\textbf{E 88. } Encontre todas as triplas de inteiros positivos $(x, y, z)$ tais
que $$ (x+y)(1+xy)= 2^{z}. $$ \source{Vietnam 2004}

\textbf{E 89. } Se $R$ e $S$ são dois retângulos com lados inteiros tais que o
perímetro de $R$ é igual à área de $S$ e o perímetro de $S$ é igual à área de
$R$, então chamamos $R$ e $S$ de par amigável de retângulos. Encontre todos os
pares amigáveis de retângulos. \source{[JDS, pp. 29]}

\textbf{E 90. } Encontre todas as soluções inteiras de $x^3 = y^2 + 4$. \source{}

%%% Local Variables:
%%% mode: latex
%%% coding: utf-8-unix
%%% fill-column: 80
%%% TeX-master: "MASTER"
%%% End:
