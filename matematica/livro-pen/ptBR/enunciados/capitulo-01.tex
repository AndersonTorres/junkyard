\chapter{Divisibility Theory}

\quoting{Por que números são belos? É como perguntar por que a Nona Sinfonia de
Beethoven é bela. Se você não consegue ver o porquê, ninguém poderá te
contar. Eu sei que números são belos. Se eles não são, nada é.}{Paul Erd\"os}

\textbf{A 1. } Mostre que se $x, y, z$ são inteiros positivos, então
$(xy+1)(yz+1)(zx+1)$ é um quadrado perfeito se e somente se $xy+1$, $yz+1$,
$zx+1$ são todos quadrados perfeitos. \source{Kiran S. Kedlaya}

\textbf{A 2. } Encontre infinitas triplas $(a, b, c)$ de inteiros positivos tais
que $a$, $b$, $c$ estão em progressão aritmética e tais que $ab+1$, $bc+1$, and
$ca+1$ são quadrados perfeitos. \source{AMM, Problem 10622, M. N. Deshpande}

\textbf{A 3. } Sejam $a$ e $b$ inteiros positivos tais que $ab + 1$ é divisor de
$a^2 + b^2$. Mostre que $$\frac{a^2 + b^2}{ab + 1}$$ é o quadrado de um número
inteiro. \source{IMO 1988/6}

\textbf{A 4. } Se $a, b, c$ são inteiros positivos tais que
$$ 0 \le a^2 +b^2 -abc \le c, $$ mostre que $a^2 +b^2 -abc$ é um quadrado
perfeito. \source{CRUX, Problem 1420, Shailesh Shirali}

\textbf{A 5. } Sejam $x$ and $y$ inteiros positivos tais que $xy$ é divisor de
$x^{2}+y^{2}+1$. Mostre que $$ \frac{x^2 +y^2 +1}{xy}=3. $$ \source{}

\textbf{A 6. }
\begin{enumerate}
\item Encontre infinitos pares de inteiros $a$ e $b$ com $1<a<b$, tais que $ab$
é divisor perfeito de $a^2 +b^2 -1$.
\item Com $a$ e $b$ como acima, quais são os possíveis valores de
  $$ \frac{a^2 +b^2 -1}{ab}? $$
\end{enumerate} \source{CRUX, Problem 1746, K. Guy and Richard J.Nowakowki}

\textbf{A 7. } Seja $n$ um inteiro positivo tal que $2+2\sqrt{28n^2 +1}$ é
inteiro. Mostre que $2+2\sqrt{28n^2 +1}$ é o quadrado de um
inteiro. \source{1969 E\"otv\"os-K\"ursch\'ak Mathematics Competition}

\textbf{A 8. } Os inteiros $a$ e $b$ têm a propriedade que para cada inteiro
não-negativo $n$ o número $2^n{a}+b$ é o quadrado de um inteiro. Prove que
$a=0$. \source{Poland 2001}

\textbf{A 9. } Prove que entre quaisquer dez inteiros positivos, ao menos um
deles é primo com o produto dos outros.\source{[IHH, pp. 211]}

\textbf{A 10. } Seja $n$ um inteiro positivo com $n \ge 3$. Mostre
$$ n^{n^{n^{n}}}-n^{n^{n}} $$ é múltiplo de $1989$.  \source{[UmDz pp.13] Unused
  Problem for the Balkan MO}

\textbf{A 11. } Sejam $a, b, c, d$ inteiros. Mostre que o produto
$$ (a-b)(a-c)(a-d)(b-c)(b-d)(c-d) $$ é múltiplo de $12$. \source{Slovenia 1995}

\textbf{A 12. } Sejam $k,m,$ and $n$ números naturais tais que $m+k+1$ é um
primo maior que $n+1$. Seja $c_{s}=s(s+1).$ Prove que o produto
$$ (c_{m+1}-c_{k})(c_{m+2}-c_{k})\cdots (c_{m+n}-c_{k}) $$ é múltiplo do produto
$c_{1}c_{2}\cdots c_{n}$. \source{Putnam 1972}

\textbf{A 13. } Mostre que para todos os números primos $p$, $$
Q(p)=\prod^{p-1}_{k=1} k^{2k-p-1} $$ é inteiro. \source{AMM, Problem E2510, Saul
Singer}

\textbf{A 14. } Seja $n$ um inteiro com $n \ge 2$. Mostre $n$ não é divisor de
$2^{n}-1$. \source{}

\textbf{A 15. } Suponha que $k \ge 2$ e $n_{1}, n_{2}, \cdots, n_{k} \ge 1$
sejam números naturais com a propriedade $$ n_{2} \; | \; 2^{n_1} -1, n_{3} \; |
\; 2^{n_2} -1, \cdots, n_{k} \; | \; 2^{n_{k-1}} -1, n_{1} \; | \; 2^{n_k}
-1. $$ Mostre que $n_{1}=n_{2}=\cdots=n_{k}=1$. \source{IMO Long List 1985 P
(RO2)}

\textbf{A 16. } Determine se existe um inteiro positivo $n$ tal que $n$ tem
exatamente $2000$ divisores primos e $2^{n}+1$ é múltiplo de $n$. \source{IMO
2000/5}

\textbf{A 17. } Sejam $m$ e $n$ números naturais tais que
$$A=\frac{(m+3)^n +1}{3m}$$ é um inteiro. Prove que $A$ é ímpar.
\source{Bulgaria 1998}

\textbf{A 18. } Sejam $m$ e $n$ números naturais e $mn+1$ um múltiplo de
$24$. Mostre que $m+n$ é múltiplo de $24$. \source{Slovenia 1994}

\textbf{A 19. } Seja $f(x)=x^3 +17$. Prove que para todo $n \ge 2$ existe um
número natural $x$ para o qual $f(x)$ é múltiplo de $3^n$ mas não de $3^{n+1}$.
\source{Japan 1999}

\textbf{A 20. } Determine todos os inteiros positivos $n$ para os quais existe
um inteiro $m$ tal que $2^{n}-1$ é divisor de $m^{2}+9$. \source{IMO Short List
1998}

\textbf{A 21. } Seja $n$ um inteiro positivo. Mostre que o produto de $n$
inteiros positivos consecutivos é divisível por $n!$ \source{}

\textbf{A 22. } Prove que o número $$ \sum_{k=0}^{n}\binom{2n+1}{2k+1}2^{3k} $$
não é múltiplo de $5$ para nenhum inteiro $n\geq 0$. \source{IMO 1974/3}

\textbf{A 23. } ({\it Teorema de Wolstenholme}) Prove que se $$
1+\frac{1}{2}+\frac{1}{3}+\cdots+\frac{1}{p-1} $$ é expresso como fração, onde
$p \ge 5$ é primo, então $p^2$ é divisor do numerador. \source{[GhEw pp.104]}

\textbf{A 24. } Seja $p>3$ um número primo e $k=\lfloor\frac{2p}{3}\rfloor
$. Prove que $$ \binom{p}{1}+\binom{p}{2}+\cdots +\binom{p}{k}$$ é múltiplo de
$p^2$. \source{Putnam 1996}

\textbf{A 25. } Mostre que $\binom{2n}{n} \; | \; \text{lcm}(1,2, \cdots, 2n)$
para todos os inteiros positivos $n$. \source{}

\textbf{A 26. } Sejam $m$ e $n$ inteiros não-negativos arbitrários. Prove que $$
\frac{(2m)!(2n)!}{m! n!(m+n)!} $$ é um inteiro. \source{IMO 1972/3}

\textbf{A 27. } Mostre que os coeficientes da expansão binomial de $(a+b)^n$,
onde $n$ é um inteiro positivo, são todos ímpares se e somente se are all odd,
if and only $n$ é da forma $2^{k}-1$ para algum inteiro positivo $k$. \source{}

\textbf{A 28. } Prove que a expressão $$ \frac{\gcd(m, n)}{n}\binom{n}{m} $$ é
um inteiro para todos os pares de inteiros positivos $(m, n)$ com $n \ge m \ge
1$. \source{Putnam 2000}

\textbf{A 29. } Para quais inteiros positivos $k$ é verdadeiro que existem
infinitos pares de inteiros positivos $(m, n)$ tais que
$$ \frac{(m+n-k)!}{m!  \; n!} $$ é inteiro? \source{AMM Problem E2623, Ivan
  Niven}

\textbf{A 30. } Mostre que se $n \ge 6$ é composto, então $n$ é divisor de
$(n-1)!$. \source{}

\textbf{A 31. } Mostre que existem infinitos inteiros positivos $n$ tais que
$n^{2}+1$ é divisor de $n!$. \source{Kazakhstan 1998}

\textbf{A 32. } Sejam $a$ e $b$ números naturais tais que
$$ \frac{a}{b}=1-\frac{1}{2}+\frac{1}{3}-\frac{1}{4}+\cdots -\frac{1}{1318}+
\frac{1}{1319}. $$ Prove que $a$ é divisível por $1979$. \source{IMO 1979/1}

\textbf{A 33. } Sejam $a,b,n\in \mathbb{N}$ com $b>1$ e tais que $b^{n}-1$ é
divisor de $a$. Mostre que na base $b$ o número $a$ tem pelo menos $n$ dígitos
diferentes de zero. \source{IMO Short List 1993}

\textbf{A 34. } Sejam $p_{1}, p_{2}, \cdots, p_{n}$ primos distintos maiores que
$3$. Mostre que $$ 2^{p_{1} p_{2} \cdots p_{n}} +1 $$ tem pelo menos $4^n$
divisores. \source{IMO Short List 2002 N3}

\textbf{A 35. } Seja $p \ge 5$ um número primo. Prove que existe um inteiro $a$
com $1 \le a \le p-2$ tal que nem $a^{p-1} -1$ nem $(a+1)^{p-1} -1$ são
múltiplos de $p^2$. \source{IMO Short List 2001 N4}

\textbf{A 36. } Sejam $n$ e $q$ inteiros com $n \ge 5$, $2 \le q \le n$. Prove
que $q-1$ é divisor de $\left\lfloor \frac{(n-1)!}{q}\right\rfloor
$. \source{Australia 2002}

\textbf{A 37. } Se $n$ é um número natural, prove que o número
$(n+1)(n+2)\cdots(n+10)$ não é um quadrado perfeito. \source{Bosnia and
Herzegovina 2002}

\textbf{A 38. } Seja $p$ um primo com $p>5$, e seja $S=\{p-n^2 \vert n \in
\mathbb{N}, {n}^{2}<p \}$. Prove que $S$ contém dois elementos $a$ e $b$ tais $a
\vert b$ e $1<a<b$. \source{MM, Problem 1438, David M. Bloom}

\textbf{A 39. } Seja $n$ um inteiro positivo. Prove que as duas afirmações a
seguir são equivalentes.

\begin{itemize}
\item $n$ não é múltiplo de $4$
\item Existem $a, b \in \mathbb{Z}$ tais que $a^{2}+b^{2}+1$ é múltiplo de $n$.
\end{itemize} \source{}

\textbf{A 40. } Determine o maior divisor comum dos elementos do conjunto
$$ \{n^{13}-n \; \vert \; n \in \mathbb{Z} \}. $$ \source{[PJ pp.110] UC
  Berkeley Preliminary Exam 1990}

\textbf{A 41. } Mostre que existem infinitos números compostos $n$ tais que
$3^{n-1}-2^{n-1}$ é múltiplo de $n$. \source{[Ae pp.137]}

\textbf{A 42. } Suponha que $2^n +1$ é um primo ímpar para algum inteiro
positivo $n$. Mostre que $n$ deve ser potência de $2$. \source{}

\textbf{A 43. } Suponha que $p$ é um número primo maior que $3$. Prove que
$7^{p}-6^{p}-1$ é múltiplo de $43$. \source{Iran 1994}

\textbf{A 44. } Suponha que $4^{n}+2^{n}+1$ é primo para algum inteiro positivo
$n$. Mostre que $n$ deve ser uma potência de $3$. \source{Germany 1982}

\textbf{A 45. } Seja $b,m,n\in\mathbb{N}$ com $b>1$ e $m\not=n$. Suponha que
$b^{m}-1$ e $b^{n}-1$ têm o mesmo conjunto de divisores primos. Mostre que $b+1$
deve ser uma potência de $2$. \source{IMO Short List 1997}

\textbf{A 46. } Sejam $a$ e $b$ inteiros. Mostre que $a$ e $b$ têm a mesma
paridade se e somente se existem inteiros $c$ e $d$ tais que $a^2 +b^2 +c^2 +1 =
d^2$. \source{Romania 1995, I. Cucurezeanu}

\textbf{A 47. } Seja $n$ um inteiro positivo com $n>1$. Prove que
$$ \frac{1}{2} + \cdots+ \frac{1}{n} $$ não é inteiro. \source{[Imv, pp. 15]}

\textbf{A 48. } Seja $n$ um inteiro positivo. Prove que
$$ \frac{1}{3} + \cdots+ \frac{1}{2n+1} $$ não é inteiro. \source{[Imv, pp. 15]}

\textbf{A 49. } Prove que não existe inteiro positivo $n$ tal que, para $k=1,
\ldots, 9$ o dígito decimal mais à esquerda de $(n+k)!$ é igual a
$k$. \source{IMO Short List 2001 N1}

\textbf{A 50. } Mostre que todo inteiro $k\ge 1$ tem um múltiplo menor que $k^4$
cuja expansão decimal tem no máximo quatro dígitos distintos. \source{Germany
2000}

\textbf{A 51. } Sejam $a,b,c$ e $d$ inteiros ímpares tais que $0<a<b<c<d$ e
$ad=bc$. Prove que se $a+d=2^{k}$ e $b+c=2^{m}$ para alguns inteiros $k$ e $m$,
então $a=1$. \source{IMO 1984/6}

\textbf{A 52. } Seja $d$ qualquer inteiro diferente de 2, 5, e 13. Mostre que
podemos encontrar $a$ e $b$ no conjunto $\{2,5,13,d\}$ tal que $ab - 1$ não seja
quadrado perfeito. \source{IMO 1986/1}

\textbf{A 53. } Suponha que $x, y,$ e $z$ são inteiros positivos com $xy=z^2
+1$. Prove que existem inteiros $a, b, c,$ and $d$ tais que $x=a^2 +b^2$, $y=c^2
+d^2$, e $z=ac+bd$. \source{Iran 2001}

\textbf{A 54. } É dito que um número natural $n$ tem a propriedade $P$, se
quando $n$ é divisor de $a^{n}-1$ para algum inteiro $a$, $n^2$ também é
necessariamente divisor de $a^{n}-1$.

\begin{enumerate}
\item Mostre que todo número primo $n$ tem a propriedade $P$.
\item Mostre que existem infinitos números compostos $n$ que possuem a
propriedade $P$.
\end{enumerate} \source{IMO ShortList 1993 IND5}

\textbf{A 55. } Mostre que para cada número natural $n$ o produto
$$ \left( 4 -\frac{2}{1} \right) \left( 4 -\frac{2}{2} \right) \left( 4
  -\frac{2}{3} \right) \cdots \left( 4 -\frac{2}{n} \right) $$ é
inteiro. \source{Czech and Slovak Mathematical Olympiad 1999}

\textbf{A 56. } Sejam $a, b$, e $c$ inteiros tais que $a+b+c$ é divisor de $a^2
+b^2 +c^2$. Prove que existem infinitos inteiros positivos $n$ tais que $a+b+c$
é divisor de $a^n +b^n +c^n$. \source{Romania 1987, L. Panaitopol}

\textbf{A 57. } Prove que para todo $n \in \mathbb{N}$ a seguinte proposição
vale: $7|3^n +n^3$ se e somente se $7|3^{n} n^3 +1$. \source{Bulgaria 1995}

\textbf{A 58. } Seja $k\ge 14$ um inteiro, e seja $p_k$ o maior número primo
estritamente menor que $k$. Você pode assumir que $p_k\ge \tfrac{3k}{4}$. Seja
$n$ um inteiro composto. Prove que

\begin{enumerate}
\item se $n=2p_k$, então $n$ não é divisor de $(n-k)!$,
\item se $n>2p_k$, então $n$ é divisor de $(n-k)!$.
\end{enumerate} \source{APMO 2003/3}

\textbf{A 59. } Suponha que $n$ tenha (pelo menos) duas representações
essencialmente distintas como soma de dois quadrados. Especificamente, seja
$n=s^{2}+t^{2}=u^{2}+v^{2}$, onde $s \ge t \ge 0$, $u \ge v \ge 0$, e
$s>u$. Mostre que $\gcd(su-tv, n)$ é um divisor próprio de $n$. \source{[AaJc,
pp. 250]}

\textbf{A 60. } Prove que existe um número infinito de pares ordenados $(a,b)$
de inteiros tais que para cada inteiro positivo $t$, o número $at+b$ é
triangular se e somente se $t$ é um número triangular. \source{Putnam 1988/B6}

\textbf{A 61. } Para qualquer inteiro positivo $n>1$, seja $p(n)$ o maior
divisor primo de $n$. Prove that existem infinitos inteiros positivos $n$ com
$$ p(n)<p(n+1)<p(n+2). $$ \source{Bulgaria 1995}

\textbf{A 62. } Seja $p(n)$ o maior divisor ímpar de $n$. Prove que
$$ \frac{1}{2^n} \sum_{k=1}^{2^n} \frac{p(k)}{k} > \frac{2}{3}. $$ \source{Germany 1997}

\textbf{A 63. } Temos uma grande pilha de cartões. Em cada cartão um dos números
$1$, $2$, $\cdots$, $n$ é escrito. É sabido que a soma de todos os números de
todos os cartões é igual a $k \cdot n!$ para algum inteiro $k$. Prove que é
possível arranjar os cartões em $k$ pilhas de forma que a soma dos números
escritos nos cartões em cada pilha é igual a $n!$. \source{[Tt] Tournament of
the Towns 2002 Fall/A-Level}

\textbf{A 64. } O último dígito decimal do número $x^2 +xy+y^2$ é zero (onde $x$
e $y$ são inteiros positivos). Prove que os dois últimos dígitos decimais deste
número são zeros. \source{[Tt] Tournament of the Towns 2002 Spring/O-Level}

\textbf{A 65. } Clara computou o produtos dos primeiros $n$ inteiros positivos,
e Valerid computou o produtos dos primeiros $m$ inteiros positivos pares, onde
$m \ge 2$. Elas obtiveram a mesma resposta. Prove que uma delas cometeu um
engano. \source{[Tt] Tournament of the Towns 2001 Fall/O-Level}

\textbf{A 66. } ({\it Teorema dos Quatro Números}) Sejam $a, b, c,$ e $d$
inteiros positivos tais que $ab=cd$. Mostre que existem inteiros positivos $p,
q, r,s$ tais que $$ a=pq, \;\; b=rs, \;\; c=ps, \;\; d=qr. $$ \source{[PeJs,
pp. 5]}

\textbf{A 67. } Suponha que $S=\{a_{1}, \cdots, a_{r}\}$ é um conjunto de
inteiros positivos, e seja $S_k$ o conjunto dos subconjuntos de $S$ com $k$
elementos. Mostre que
$$ \text{lcm}(a_{1}, \cdots, a_{r})=\prod_{i=1}^r \prod_{s\in
  S_i}\gcd(s)^{\left((-1)^i\right)}.$$ \source{[Her, pp. 14]}

\textbf{A 68. } Prove que se o primo ímpar $p$ é divisor de $a^{b}-1$, onde $a$
e $b$ são inteiros positivso, então $p$ aparece com a mesma potência na
fatoração de $b(a^{d}-1)$, onde $d=\gcd(b,p-1)$. \source{MM, June 1986, Problem
1220, Gregg Partuno}

\textbf{A 69. } Suponha que $m=nq$, onde $n$ e $q$ são inteiros positivos. Prove
que a soma de coeficientes binomiais
$$ \sum_{k=0}^{n-1} \binom{\gcd(n, k)q}{\gcd(n, k)} $$ é múltipla de
$m$. \source{MM, Sep. 1984, Problem 1175}

\textbf{A 70. } Determine todos os inteiros $n > 1$ tais que $$\frac{2^n +
1}{n^2}$$ é um inteiro. \source{IMO 1990/3 (ROM5)}

\textbf{A 71. } Determine todos os pares $(n,p)$ de inteiros não-negativos tais
que

\begin{itemize}
\item $p$ é primo,
\item $n<2p$,
\item $(p-1)^{n} + 1$ é múltiplo de $n^{p-1}$.
\end{itemize} \source{IMO 1999/4}

\textbf{A 72. } Determine todos os pares $(n,p)$ de inteiros não-negativos tais
que

\begin{itemize}
\item $p$ é primo,
\item $n \ge 1$,
\item $(p-1)^{n} + 1$ é múltiplo de $n^{p-1}$.
\end{itemize} \source{}

\textbf{A 73. } Encontre um inteiro $n$, onde $100 \leq n \leq 1997$, tal
que $$\frac{2^n+2} {n}$$ também é um inteiro. \source{APMO 1997/2}

\textbf{A 74. } Encontre todas as triplas $(a,b,c)$ de inteiros positivos tais
que $2^{c}-1$ é divisor de $2^{a}+2^{b}+1$. \source{APMC 2002}

\textbf{A 75. } Encontre todos os inteiros $\,a,b,c\,$ com $\,1<a<b<c\,$ tais
que $(a-1)(b-1)(c-1)$ é divisor de $abc-1$. \source{IMO 1992/1}

\textbf{A 76. } Encontre todos os inteiros positivos representáveis de maneira
única como $$\frac{x^2 + y}{xy + 1},$$ onde $x$ e $y$ são inteiros positivos.
\source{Russia 2001}

\textbf{A 77. } Determine todos os pares ordenados $(m, n)$ de inteiros
positivos tais que $$ \frac{n^3 + 1}{mn - 1} $$ é um inteiro. \source{IMO
1994/4}

\textbf{A 78. } Determine todos os pares de inteiros $(a, b)$ tais que
$$ \frac{a^2}{2ab^2 - b^3 +1} $$ é um inteiro positivo. \source{IMO 2003/2}

\textbf{A 79. } Encontre todos os pares de inteiros positivos $m, n \ge 3$ para
os quais existem infinitos inteiros positivos $a$ such tais que
$$ \frac{a^m +a-1}{a^n +a^2 -1} $$ é ele mesmo inteiro. \source{IMO 2002/3}

\textbf{A 80. } Determine todas as triplas de inteiros positivos $(a, m, n)$
tais que $a^m +1$ é divisor de $(a+1)^n$. \source{IMO Short List 2000 N4}

\textbf{A 81. } Que inteiros podem ser representados
por $$\frac{(x+y+z)^2}{xyz}$$ onde $x$, $y$, e $z$ são inteiros positivos?
\source{AMM, Problem 10382, Richard K. Guy}

\textbf{A 82. } Encontre todos os $n \in \mathbb{N}$ tais que $ \lfloor
\sqrt{n}\rfloor$ é divisor de $n$. \source{[Tma pp. 73]}

\textbf{A 83. } Determine todos os $n \in \mathbb{N}$ para os quais

\begin{itemize}
\item $n$ não é o quadrado de nenhum inteiro,
\item $\lfloor \sqrt{n}\rfloor ^3$ é divisor de $n^2$.
\end{itemize} \source{India 1989}

\textbf{A 84. } Encontre todos os $n \in \mathbb{N}$ tais que $ 2^{n-1}$ é
divisor de $n!$. \source{[ElCr pp. 11]}

\textbf{A 85. } Encontre todos os inteiros positivos $(x, n)$ tais que
$x^{n}+2^{n}+1$ é divisor de $x^{n+1}+2^{n+1}+1$. \source{Romania 1998}

\textbf{A 86. } Encontre todos os inteiros positivos $n$ tais que $3^{n}-1$ é
múltiplo de $2^n$. \source{Romania 2005}

\textbf{A 87. } Encontre todos os inteiros positivos $n$ tais que $9^{n}-1$ é
múltiplo de $7^n$. \source{}

\textbf{A 88. } Determine todos os pares $(a, b)$ de inteiros para os quais
$a^{2}+b^{2}+3$ é múltiplo de $ab$. \source{Turkey 1994}

\textbf{A 89. } Determine todos os pares $(x, y)$ de inteiros positivos com $y
\vert x^2 +1$ e $x \vert y^3 +1$. \source{Mediterranean Mathematics Competition
2002}

\textbf{A 90. } Determine todos os pares $(a, b)$ de inteiros positivos tais que
$ab^2+b+7$ é divisor de $a^2 b+a+b$. \source{IMO 1998/4}

\textbf{A 91. } Sejam $a$ e $b$ inteiros positivos. Quando $a^{2}+b^{2}$ é
dividido por $a+b$, o quociente é $q$ e o resto é $r$. Encontre todos os pares
$(a,b)$ tais que $q^{2}+r=1977$. \source{IMO 1977/5}

\textbf{A 92. } Encontre o maior inteiro positivo $n$ tal que $n$ é múltiplo de
todos os inteiros positivos menores que $\sqrt[3]{n}$. \source{APMO 1998}

\textbf{A 93. } Encontre todos os $n \in \mathbb{N}$ tais que $3^{n}-n$ é
múltiplo de $17$. \source{}

\textbf{A 94. } Suponha que $a$ e $b$ são números naturais tais que $$
p=\frac{b}{4}\sqrt{\frac{2a-b}{2a+b}} $$ é um número primo. Qual é o maior valor
possível de $p$? \source{Iran 1998}

\textbf{A 95. } Encontre todos os inteiros positivos $n$ que têm exatamente $16$
divisores inteiros positivos $d_{1},d_{2} \cdots, d_{16}$ tais que
$1=d_{1}<d_{2}<\cdots<d_{16}=n$, $d_6=18$, e $d_{9}-d_{8}=17$. \source{Ireland
1998}

\textbf{A 96. } Suponha que $n$ é um inteiro positivo, e sejam
$$ d_{1}<d_{2}<d_{3}<d_{4} $$ os quatro menores inteiros positivos divisores de
$n$. Encontre todos os inteiros $n$ tais que
$$ n={d_1}^{2}+{d_2}^{2}+{d_3}^{2}+{d_4}^{2}. $$ \source{Iran 1999}

\textbf{A 97. } Seja $n$ um inteiro positivo com $k \ge 22$ divisores $1=d_{1} <
d_{2} < \cdots < d_{k} =n$, todos diferentes. Determine todos os $n$ tais que $$
{d_{7}}^2 +{d_{10}}^2 = \left( \frac{n}{d_{22}} \right)^2. $$ \source{Belarus
1999, I. Voronovich}

\textbf{A 98. } Seja $n \ge 2$ um inteiro positivo, com divisores
$$ 1=d_{1} < d_{2} < \cdots < d_{k} =n \;. $$ Prove que
$$ d_{1}d_{2}+d_{2}d_{3}+\cdots+d_{k-1}d_{k} $$ é sempre menor que $n^2$, e
determine quando este valor é divisor de $n^2$. \source{IMO 2002/4}

\textbf{A 99. } Encontre todos os inteiros positivos $n$ tais que $n$ tem
exatamente $6$ divisores positivos $1<d_{1}<d_{2}<d_{3}<d_{4}<n$ e
$1+n=5(d_{1}+d_{2}+d_{3}+d_{4})$. \source{Singapore 1997}

\textbf{A 100. } Encontre todos os números compostos $n$ tendo a propriedade que
todo divisor $d$ de $n$ é tal que $n-20 \le d \le n-12$. \source{Belarus 1998,
E. Barabanov, I. Voronovich}

\textbf{A 101. } Determine todos os números de três dígitos $N$ com a
propriedade que $N$ é múltiplo de $11$ e $\frac{N}{11}$ é igual à soma dos
quadrados dos dígitos de $N$. \source{IMO 1960/1}

\textbf{A 102. } Quando $4444^{4444}$ é escrito na notação decimal, a soma de
seus dígitos é $A$. Seja $B$ a soma dos dígitos de $A$. Encontre a soma dos
dígitos de $B$. ($A$ e $B$ são escritos em notação decimal.) \source{IMO 1975/4}

\textbf{A 103. } Um número esburacado é um inteiro positivo cujos dígitos
decimais são alternadamente zero e não-zero, e o dígito das unidades não é
nulo. Determine todos os inteiros positivos que não são divisores de nenhum
número esburacado. \source{IMO Short List 1994 N7}

\textbf{A 104. } Encontre o menor inteiro positivo $n$ tal que

\begin{itemize}
\item $n$ tem exatamente $144$ divisores positivos distintos,
\item existem dez inteiros consecutivos distintos entre os divisores positivos
de $n$.
\end{itemize} \source{IMO Long List 1985 (TR5)}

\textbf{A 105. } Determine o menor valor possível do número natural $n$ tal que
$n!$ termine em exatamente $1987$ zeros. \source{IMO Long List 1987}

\textbf{A 106. } Encontre quatro inteiros positivos, todos eles menores ou
iguais a $70000$ e cada um tendo mais de $100$ divisores. \source{IMO Short List
1986 P10 (NL1)}

\textbf{A 107. } Para cada inteiro $n>1$, seja $p(n)$ o mair fator primo de
$n$. Determine todas as triplas $(x, y, z)$ de inteiros positivos distintos
satisfazendo

\begin{itemize}
\item $x, y, z$ estão em progressão aritmética
\item $p(xyz) \le 3$.
\end{itemize} \source{British Mathematical Olympiad 2003, 2-1}

\textbf{A 108. } Encontre todos os inteiros positivos $a$ e $b$ tais
que $$\frac{a^2+b}{b^2-a}\text{ e }\frac{b^2+a}{a^2-b}$$ são ambos
inteiros. \source{APMO 2002/2}

\textbf{A 109. } Para cada inteiro positivo $n$, escreva a soma $\sum_{m=1}^n
1/m$ na forma $p_n/q_n$, onde $p_n$ e $q_n$ são inteiros positivos primos entre
si.  Determine todos os $n$ tais que $5$ não é divisor de $q_n$. \source{Putnam
1997/B3}

\textbf{A 110. } Encontre todos os números naturais $n$ tais que o número
$n(n+1)(n+2)(n+3)$ tenha exatamente três divisores primos distintos.
\source{Spain 1993}

\textbf{A 111. } Prove que existem infinitos pares $(a, b)$ de inteiros
positivos primos entre si tais que
$$ \frac{a^2 -5}{b} \;\; \text{e} \;\; \frac{b^2 -5}{a} $$ são ambos inteiros
positivos. \source{Germany 2003}

\textbf{A 112. } Encontre todas as triplas $(l, m, n)$ de inteiros positivos
distintos satisfazendo
$$ {\gcd(l, m)}^2 = l+m, \; {\gcd(m, n)}^2 = m+n, \; \text{e} \;\; {\gcd(n,
  l)}^2 = n+l. $$ \source{Russia 1997}

\textbf{A 113. } Qual é o maior divisor comum do conjunto de números
$$ \{ {16}^n + 10n-1 \; \vert \; n=1,2,\cdots \}? $$ \source{[EbMk, pp. 16]}

\textbf{A 114. } Existe um inteiro de quatro dígitos (em forma decimal) tal que
nenhuma substituição de três de seus dígitos por quaisquer outros três resulte
em um múltiplo de $1992$? \source{[Ams, pp. 102], I. Selishev}

\textbf{A 115. } Qual é o menor inteiro positivo que consiste de cada um dos dez
dígitos na base 10, cada um usado uma única vez, e múltiplo de cada um dos
dígitos de 2 até 9? \source{[JDS, pp. 27]}

\textbf{A 116. } Encontre o menor inteiro positivo $n$ tal que
$$ 2^{1989} \; \vert \; m^n -1 $$ para todos os inteiros positivos ímpares
$m>1$. \source{[Rh2, pp. 98]}

\textbf{A 117. } Determine a maior potência de $1980$ que é divisora de
$$ \frac{(1980n)!}{(n!)^{1980}}. $$ \source{MM, Jan. 1981, Problem 1089, M. S. Klamkin}

\textbf{A 118. } Encontre $\gcd\{a^4-b^4|a,b\in P\}$, onde $P$ é o conjunto dos
primos com pelo menos dois dígitos. \source{Flanders 1990}

\textbf{A 119. } Sejam $a,b$ dois inteiros positivos e $an$ and $bn$ tendo as
mesmas somas de dígitos para todo $n\in\mathbb{N}$. Prove que $\frac ab=10^k$
para algum inteiro $k$. \source{Harazi \& Dzeta from MathLinks}

\textbf{A 120. } Seja $b>5$ Um inteiro. Para cada inteiro positivo $n$,
defina $$x_n=\underbrace{11\cdots1}_{n-1}\underbrace{22\cdots2}_{n}5,$$ escrito
em base $b$. Prove que a seguinte condição ocorre se e somente se $b=10$:
\textit{existe um inteiro positivo $M$ tal que para todo inteiro $n>M$, $x_n$ é
um quadrado perfeito.} \source{IMO Shortlist 2003}

%%% Local Variables: %%% mode: latex %%% coding: utf-8-unix %%% fill-column: 80
%%% TeX-master: "MASTER" %%% End:
