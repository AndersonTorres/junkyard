\chapter{Teoria Algébrica dos Números}

\textbf{J 1. } Seja $n$ um inteiro positivo. Mostre que existem infinitos primos
$p$ tais que a menor raiz primitiva de $p$ é maior que $n$. \source{}

\textbf{J 2. } Mostre que para cada primo ímpar $p$, existe um inteiro $g$ tal
que $1<g<p$ e $g$ é uma raiz primitiva módulo $p^n$ para cada inteiro positivo
$n$. \source{MM, Problem 1419, William P. Wardlaw}

\textbf{J 3. } Seja $g$ uma raiz primitiva de Fibonacci $\pmod{p}$. i.e. $g$ é
uma raiz primitiva $\pmod{p}$ satisfazendo $g^2  \equiv g+1\; \pmod{p}$. Prove
que

\begin{enumerate}
\item $g-1$ também é uma raiz primitiva $\pmod{p}$.
\item se $p=4k+3$ então $(g-1)^{2k+3} \equiv g-2 \pmod{p}$, e deduza que $g-2$
  também é uma raiz primitiva $\pmod{p}$.
\end{enumerate} \source{[Km, Problems Sheet 3-9]}

\textbf{J 4. } Seja $p$ um primo ímpar. Se $g_{1}, \cdots, g_{\phi(p-1)}$ são
raízes primitivas $\pmod{p}$ no intervalo $1<g \le p-1$, prove que
$$ \sum_{i=1}^{\phi(p-1)} g_i \equiv \mu(p-1) \pmod{p}. $$ \source{[Km, Problems
  Sheet 3-11]}

\textbf{J 5. } Suponha que $m$ não tenha raiz primitiva. Mostre que
$$ a^{ \frac{\phi(m)}{2}} \equiv 1 \; \pmod{m} $$ para todo $a$ primo com $m$.
\source{[AaJc, pp. 178]}

\textbf{J 6. } Suponha que $p>3$ é primo. Prove que o produto das raízes
primitivas de $p$ entre $1$ e $p-1$ é congruente a $1$ módulo $p$.
\source{[AaJc, pp. 181]}

\textbf{J 7. } Encontre todos os inteiros positivos $n$ que são resíduos
quadráticos módulo todos os primos maiores que $n$. \source{CRUX, Problem 2344,
  Murali Vajapeyam}

\textbf{J 8. } Os inteiros positivos $a$ e $b$ são tais que os números $15a+16b$
e $16a-15b$ são ambos quadrados de inteiros. Qual é menor valor possível que o
menor destes quadrados pode assumir? \source{IMO 1996/4}

\textbf{J 9. } Seja $p$ um primo ímpar. Mostre que o menor resíduo
não-quadrático positivo de $p$ é menor que $\sqrt{p}+1$. \source{[IHH pp.147]}

\textbf{J 10. } Seja $M$ um inteiro, e seja $p$ um primo, com $p>25$. Mostre que
o conjunto $\{M, M+1, \cdots, M+ 3\lfloor \sqrt{p} \rfloor -1\}$ contém um
resíduo não-quadrático módulo $p$. \source{[Imv, pp. 72]}

\textbf{J 11. } Seja $p$ um primo ímpar e $Z_p$ denote o corpo (the field of)
dos inteiros módulo $p$. Quantos elementos há no conjunto
$$ \{x^2: x \in Z_p\} \cap \{y^2 + 1: y \in Z_p\}? $$ \source{Putnam 1991/B5}

\textbf{J 12. } Sejam $a, b, c$ inteiros e seja $p$ um primo ímpar com
$$ p \not\vert a \;\; \text{and} \;\; p \not\vert b^2 -4ac. $$ Mostre que
$$ \sum_{k=1}^{p} \left( \frac{ak^2 +bk+c}{p} \right) = - \left( \frac{a}{p}
\right). $$ \source{[Ab, pp. 34]}

\textbf{J 13. } Suponha que $p(x) \in \mathbb{Z}[x]$ and $P(a)P(b)=-(a-b)^2$ ara
algum par $a, b \in \mathbb{Z}$ distintos. Prove that $P(a)+P(b)=0$. \source{MM,
  Problem Q800, Bjorn Poonen}

\textbf{J 14. } Prove que não existe polinômio não-constante $f(x)$ com
coeficientes inteiros tal que $f(n)$ é primo para todo $n \in
\mathbb{N}$. \source{}

\textbf{J 15. } Seja $n \ge 2$ um inteiro. Prove que se $k^2 + k + n$ é primo
para todos os inteiros $k$ tais que $0 \leq k \leq \sqrt{\frac{n}{3}}$, então
$k^2 +k + n$ é primo para todos os inteiros $k$ tais que $0 \leq k \leq n -
2$. \source{IMO 1987/6}

\textbf{J 16. } Um primo $p$ tem dígitos decimais $p_{n}p_{n-1} \cdots p_0$ com
$p_{n}>1$. Mostre que o polinômio
$p_{n}x^{n} + p_{n-1}x^{n-1}+\cdots+ p_{1}x + p_0$ não pode ser representado
como produto de dois polinômios não constantes de coeficientes
inteiros. \source{Balkan Mathematical Olympiad 1989}

\textbf{J 17. } ({\it Critério de Eisentein}) Seja
$f(x)=a_{n}x^{n} +\cdots +a_{1}x+a_{0}$ um polinômio não constante de
coeficientes inteiros. Se existe um primo $p$ tal que $p$ é divisor de cada um
dos $a_{0}$, $a_{1}$, $\cdots$,$a_{n-1}$ mas $p$ não é divisor de $a_{n}$ e
$p^2$ não é divisor de $a_{0}$, ento $f(x)$ é irredutível em
$\mathbb{Q}[x]$. \source{[Twh, pp. 111]}

\textbf{J 18. } Prove que para um primo $p$, $x^{p-1}+x^{p-2}+ \cdots +x+1$ é
irredutível em $\mathbb{Q}[x]$. \source{Gauss, [Twh, pp. 114]}

\textbf{J 19. } Seja $f(x)=x^{n}+5x^{n-1}+3$, onde $n>1$ é um inteiro. Prove que
$f(x)$ não pode ser expressado como produto de dois polinômios não constantes de
coeficientes inteiros. \source{IMO 1993/1}

\textbf{J 20. } Mostre que um polinômio de grau ímpar $2m+1$ sobre
$\mathbb{Z}$, $$ f(x)=c_{2m+1}x^{2m+1} +\cdots +c_{1}x+c_{0}, $$ é irredutível
se existe um primo $p$ tal que
$$ p \not\vert c_{2m+1}, p \vert c_{m+1}, c_{m+2}, \cdots, c_{2m}, p^2 \vert
c_{0}, c_{1}, \cdots, c_{m}, \; \text{and} \; p^3 \not\vert c_{0}. $$
\source{(Eugen Netto) [Ac, pp. 87] Para uma prova, veja [En].}

\textbf{J 21. } Para inteiros não-negativos $n$ e $k$, seja $P_{n, k}(x)$ a
função racional
$$ \frac{(x^n -1)(x^n -x) \cdots (x^n -x^{k-1})}{(x^k -1)(x^k -x) \cdots (x^k
  -x^{k-1})}. $$ Mostre que $P_{n, k}(x)$ é na realidade um polinômio para todo
$n, k \in \mathbb{N}$. \source{CRUX, Problem A230, Naoki Sato}

\textbf{J 22. } Suponha que os inteiros $a_{1}$, $a_{2}$, $\cdots$, $a_{n}$ são
distintos. Mostre que
$$ (x-a_{1})(x-a_{2}) \cdots (x-a_{n})-1 $$ não pode ser expressado como produto
de dois polinômios não constantes de coeficientes inteiros. \source{[Ae,
pp. 257]}

\textbf{J 23. } Prove que se os inteiros $a_{1}$, $a_{2}$, $\cdots$, $a_{n}$ são
todos distintos, então o polinômio
$$ (x-a_{1})^2 (x-a_{2})^2 \cdots (x-a_{n})^2 +1 $$ não pode ser expressado como produto
de dois polinômios não constantes de coeficientes inteiros. \source{[DNI, 47]}

\textbf{J 24. } Para $n>1$, sejam $a_1,a_2,..,a_n$ inteiros positivos distintos
e seja $p_i= \prod^n_{j=1, j\not=i}(a_i-a_j)$. Mostre que para todos
$k\in\mathbb{N}$ temos que $\sum_{i=1}^n \frac{a_i^k}{p_i}$ é um inteiro.
\source{UK 1981}

%%% Local Variables:
%%% mode: latex
%%% coding: utf-8-unix
%%% fill-column: 80
%%% TeX-master: "MASTER"
%%% End:
